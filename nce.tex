% Created 2016-04-21 四 09:47
\documentclass[11pt]{article}
\usepackage[utf8]{inputenc}
\usepackage[T1]{fontenc}
\usepackage{fixltx2e}
\usepackage{graphicx}
\usepackage{longtable}
\usepackage{float}
\usepackage{wrapfig}
\usepackage{soul}
\usepackage{textcomp}
\usepackage{marvosym}
\usepackage{wasysym}
\usepackage{latexsym}
\usepackage{amssymb}
\usepackage{hyperref}
\tolerance=1000
\providecommand{\alert}[1]{\textbf{#1}}

\title{New Concept English}
\author{KISS}
\date{\today}
\hypersetup{
  pdfkeywords={},
  pdfsubject={},
  pdfcreator={Emacs Org-mode version 7.9.3f}}

\begin{document}

\maketitle

\setcounter{tocdepth}{3}
\tableofcontents
\vspace*{1cm}


\section{NCE 3}
\label{sec-1}
\subsection{A puma at large}
\label{sec-1-1}

Pumas are large, cat-like animals which are found in America. When reports came into London Zoo that a wild puma had been spotted forty-five miles south of London, they were not taken seriously. However, as the evidence began to accumulate, experts from the Zoo felt obliged to investigate, for the descriptions given by people who claimed to have seen the puma were extraordinarily similar.

The hunt for the puma began in a small village where a woman picking blackberries saw ``a large cat'' only five yards away from her. It immediately ran away when she saw it, and experts confirmed that a puma will not attack a human being unless it is cornered. The search proved difficult, for the puma was often observed at one place in the morning and at another place twenty miles away in the evening. Wherever it went, it left behind it a trail of dead deer and small animals like rabbits.

Paw prints were seen in a number of places and puma fur was found clinging to bushes. Several people complained of ``cat-like noises'' at night and a businessman on a fishing trip saw the puma up a tree. The experts were now fully convinced that the animal was a puma, but where had it come from ? As no pumas had been reported missing from any zoo in the country, this one must have been in the possession of a private collector and somehow managed to escape. The hunt went on for several weeks, but the puma was not caught. It is disturbing to think that a dangerous wild animal is still at large in the quiet countryside.
\subsection{Thirteen equals one}
\label{sec-1-2}

Our vicar is always raising money for one cause or another, but he has never managed to get enough money to have the church clock repaired. The big clock which used to strike the hours day and night was damaged many years ago and has been silent ever since.

One night, however, our vicar woke up with a start: the clock was striking the hours! Looking at his watch, he saw that it was one o'clock, but the bell struck thirteen times before it stopped. Armed with a torch, the vicar went up into the clock tower to see what was going on. In the torchlight, he caught sight of a figure whom he immediately recognized as Bill Wilkins, our local grocer.

``Whatever are you doing up here Bill?'' asked the vicar in surprise.

``I'm trying to repair the bell,'' answered Bill. ``I've been coming up here night after night for weeks now. You see, I was hoping to give you a surprise.''

``You certainly did give me a surprise!'' said the vicar. ``You've probably woken up everyone in the village as well. Still, I'm glad the bell is working again.''

``That's the trouble, vicar,'' answered Bill. ``It's working all right, but I'm afraid that at one o'clock it will strike thirteen times and there's nothing I can do about it.''

``We'll get used to that Bill,'' said the vicar. ``Thirteen is not as good as one but it's better than nothing. Now let's go downstairs and have a cup of tea.''
\subsection{An unknown goddess}
\label{sec-1-3}

Some time ago, an interesting discovery was made by archaeologists on the Aegean (adj.爱琴海的; n.) island of Kea. An American team explored a temple which stands in an ancient city on the promontory of Ayia Irini. The city at one time must have been prosperous, for it enjoyed a high level of civilization. Houses---often three storeys high---were built of stone. They had large rooms with beautifully decorated walls. The city was even equipped with a drainage system, for a great many clay pipes were found beneath the narrow streets.

The temple which the archaeologists explored was used as a place of worship from the fifteenth century B.C. until Roman times. In the most sacred room of the temple, clay fragments of fifteen statues were found. Each of these represented a goddess and had, at one time, been painted. The body of one statue was found among remains dating from the fifteenth century B.C. Its missing head happened to be among remains of the fifth century B.C. This head must have been found in Classical times and carefully preserved. It was very old and precious even then. When the archaeologists reconstructed the fragments, they were amazed to find that the goddess turned out to be a very modern-looking woman. She stood three feet high and her hands rested on her hip. She was wearing a full-length skirt which swept the ground. Despite her great age, she was very graceful indeed, but, so far, the archaeologists have been unable to discover her identity.
\subsection{The double life of Alfred Bloggs}
\label{sec-1-4}

These days, people who do manual work often receive far more money than clerks who work in offices. People who work in offices are frequently referred to as'' white collar workers'' for the simple reason that they usually wear a collar and tie to go to work. Such is human nature, that a great many people are often willing to sacrifice higher pay for the privilege of becoming white collar workers. This can give rise to curious situations, as it did in the case of Alfred Bloggs who worked as a dustman for the Ellesmere Corporation.

When he got married, Alf was too embarrassed to say anything to his wife about his job. He simply told her that he worked for the Corporation. Every morning, he left home dressed in a smart black suit. He then changed into overalls (n.工作服) and spent the next eight hours as a dustman. Before returning home at night, he took a shower and changed back into his suit. Alf did this for over two years and his fellow dustmen kept his secret. Alf's wife has never discovered that she married a dustman and she never will, for Alf has just found another job. He will soon be working in an office as a junior clerk. He will be earning only half as much as he used to, but he feels that his rise in status is well worth the loss of money. From now on, he will wear a suit all day and others will call him ``Mr. Bloggs'', not ``Alf''.
\subsection{The facts}
\label{sec-1-5}

Editors of newspapers and magazines often go to extremes to provide their readers with unimportant facts and statistics. Last year a journalist had been instructed by a well-known magazine to write an article on the president's palace in a new African republic. When the article arrived, the editor read the first sentence and then refused to publish it. The article began: ``Hundreds of steps lead to the high wall which surrounds the president's palace.'' The editor at once sent the journalist a fax instructing him to find out the exact number of steps and the height of the wall.

The journalist immediately set out to obtain these important facts, but he took a long time to send them. Meanwhile, the editor was getting impatient, for the magazine would soon go to press. He sent the journalist two urgent telegrams, but received no reply. He sent yet another telegram informing the journalist that if he did not reply soon he would be fired. When the journalist again failed to reply, the editor reluctantly published the article as it had originally been written. A week later, the editor at last received a telegram from the journalist. Not only had the poor man been arrested, but he had been sent to prison as well. However, he had at last been allowed to send a cable in which he informed the editor that he had been arrested while counting the 1084 steps leading to the 15-foot wall which surrounded the president's palace.
\subsection{Smash-and-grab}
\label{sec-1-6}

The expensive shops in a famous arcade near Piccadilly were just opening. At this time of the morning, the arcade was almost empty. Mr Taylor, the owner of a jewellery shop was admiring a new window display. Two of his assistants had been working busily since 8 o'clock and had only just finished. Diamond necklaces and rings had been beautifully arranged on a background of black velvet. After gazing at the display for several minutes, Mr Taylor went back into his shop.

The silence was suddenly broken when a large car, with its headlights on and its horn blaring, roared down the arcade. It came to a stop outside the jeweler's. One man stayed at the wheel while two others with black stockings over their faces jumped out and smashed the window of the shop with iron bars. While this was going on, Mr Taylor was upstairs. He and his staff began throwing furniture out of the window. Chairs and tables went flying into the arcade. One of the thieves was struck by a heavy statue, but he was too busy helping himself to diamonds to notice any pain. The raid was all over in three minutes, for the men scrambled back into the car and it moved off at a fantastic speed. Just as it was leaving, Mr Taylor rushed out and ran after it throwing ashtrays and vases, but it was impossible to stop the thieves. They had got away with thousands of pounds worth of diamonds.
\subsection{Mutilated ladies}
\label{sec-1-7}

Children often have far more sense than their elders. This simple truth was demonstrated rather dramatically during a civil defence exercise in a small town in Canada. Most of the inhabitants were asked to take part in the exercise during which they had to pretend that their city had been bombed. Air-raid warnings were sounded and thousands of people went into special air-raid shelters. Doctors and nurses remained above ground while Police patrolled the streets in case anyone tried to leave the shelters too soon.

The police did not have much to do because the citizens took the exercise seriously. They stayed underground for twenty minutes and waited for the siren to sound again. On leaving the air-raid shelters, they saw that doctors and nurses were busy. A great many people had volunteered to act as casualties. Theatrical make-up and artificial blood had been used to make the injuries look realistic. A lot of People were lying ``dead'' in the streets. The living helped to carry the dead and wounded to special stations. A Child of six was brought in by two adults. The child was supposed to be dead. With theatrical make-up on his face, he looked as if he had died of shock. Some people were so moved by the sight that they began to cry. However, the child suddenly sat up and a doctor asked him to comment on his death. The child looked around for a moment and said, ``I think they're all crazy!''
\subsection{A famous monastery}
\label{sec-1-8}

The Great St Bernard Pass connects Switzerland to Italy. At 2470 metres, it is the highest mountain pass in Europe. The famous monastery of St Bernard, which was founded in the eleventh century, lies about a mile away. For hundreds of years, St Bernard dogs have saved the lives of travellers crossing the dangerous Pass. These friendly dogs, which were first brought from Asia, were used as watch-dogs even in Roman times. Now that a tunnel has been built through the mountains, the Pass is less dangerous, but each year, the dogs are still sent out into the snow whenever a traveller is in difficulty. Despite the new tunnel, there are still a few people who rashly attempt to cross the Pass on foot.

During the summer months, the monastery is very busy, for it is visited by thousands of people who cross the Pass in cars, As there are so many people about, the dogs have to be kept in a special enclosure. In winter, however, life at the monastery is quite different. The temperature drops to -30 and very few people attempt to cross the Pass. The monks Prefer winter to summer for they have more privacy. The dogs have greater freedom, too, for they are allowed to wander outside their enclosure. The only regular visitors to the monastery in winter are parties of skiers who go there at Christmas and Easter. These young people, who love the peace of the mountains, always receive a warm.

Welcome at St Bernard's monastery.
\subsection{Flying cats}
\label{sec-1-9}

Cats never fail to fascinate human beings. They can be friendly and affectionate towards humans, but they lead mysterious lives of their own as well. They never become submissive like dogs and horses. As a result, humans have learned to respect feline independence. Most cats remain suspicious of humans all their lives. One of the things that fascinates us most about cats is the popular belief that they have nine lives. Apparently, they is a good deal of truth in this idea. A cat's ability to survive falls is based on fact.

Recently the New York Animal Medical Centre made a study of 132 cats over a period of five months. All these cats had one experience in common: they had fallen off high buildings, yet only eight of them died from shock or injuries. Of course, New Yorkis the ideal place for such an interesting study, because there is no shortage of tall buildings. There are plenty of high-rise windowsills to fall from! One cat, Sabrina, fell 32 storeys, yet only suffered from a broken tooth. ``Cats behave like well-trained paratroopers,'' a doctor said. It seems that the further cats fall, the less they are likely to injure themselves. In a long drop, they reach speeds of 60 miles an hour and more. At high speeds, falling cats have time to relax. They stretch out their legs like flying squirrel. This increases their air-resistance and reduces the shock of impact when they hit the ground.
\subsection{The loss of Titanic}
\label{sec-1-10}

The great ship, Titanic, sailed for New York from Southampton on April 10th, 1912. She was carrying 1316 passengers and a crew of 89l. Even by modern standards, the 46,000 ton Titanic was a colossal ship. At that time, however, she was not only the largest ship that had ever been built, but was regarded as unsinkable, for she had sixteen water-tight compartments. Even if two of these were flooded, she would still be able to float. The tragic sinking of this great liner will always be remembered, for she went down on her first voyage with heavy loss of life.

Four days after setting out, while the Titanic was sailing across the icy waters of the North Atlantic, a huge iceberg was suddenly spotted by a look-out. After the alarm had been given, the great ship turned sharply to avoid a direct collision. The Titanic turned just in time, narrowly missing the immense wall of ice which rose over 100 feet out of the water beside her. Suddenly, there was a slight trembling sound from below, and the captain went down to see what had happened. The noise had been so faint that no one thought that the ship had been damaged. Below, the captain realized to his horror that the Titanic was sinking rapidly, for five of her sixteen water-tight compartments had already been flooded ! The order to abandon ship was given and hundreds of people plunged into the icy water. As there were not enough life-boats for everybody, 1500 lives were lost.
\subsection{Not guilty}
\label{sec-1-11}

Going through the Customs is a tiresome business. The strangest thing about it is that really honest people are often made to feel guilty. The hardened professional smuggler, on the other hand, is never troubled by such feelings, even if he has five hundred gold watches hidden in his suitcase. When I returned from abroad recently, a particularly officious young Customs Officer clearly regarded me as a smuggler.

``Have you anything to declare?'' he asked, looking me in the eye.

``No,'' I answered confidently.

``Would you mind unlocking this suitcase please?''

``Not at all,'' I answered. 

The Officer went through the case with great care. All the things I had packed so carefully were soon in a dreadful mess. I felt sure I would never be able to close the case again. Suddenly, I saw the Officer's face light up. He had spotted a tiny bottle at the bottom of my case and he pounced on it with delight. 

``Perfume, eh?'' he asked sarcastically. ``You should have declared that. Perfume is not exempt from import duty.''

``But it isn't perfume,'' I said. ``It's hair-oil.'' Then I added with a smile, ``It's a strange mixture I make myself.'' As I expected, he did not believe me.

``Try it!'' I said encouragingly.

The Officer unscrewed the cap and put the bottle to his nostrils. He was greeted by an unpleasant smell which convinced him that I was telling the truth. A few minutes later, I was able to hurry away with precious chalk-marks on my baggage. 
\subsection{Life on a desert island}
\label{sec-1-12}

Most of us have formed an unrealistic picture of life on a desert island. We sometimes imagine a desert island to be a sort of paradise where the sun always shines. Life there is simple and good.

Ripe fruit falls from the trees and you never have to work. The other side of the picture is quite the opposite. Life on a desert island is wretched. You either starve to death or live like Robinson Crusoe, waiting for a boat which never comes. Perhaps there is an element of truth in both these pictures, but few of us have had the opportunity to find out. 

Two men who recently spent five days on a coral island wished they had stayed there longer. They were taking a badly damaged boat from the Virgin Islands to Miami to have it repaired. During the journey, their boat began to sink. They quickly loaded a small rubber dinghy with food, matches, and tins of beer and rowed for a few miles across the Caribbean until they arrived at a tiny coral island. There were hardly any trees on the island and there was no water, but this did not prove to be a problem. The men collected rain-water in the rubber dinghy. As they had brought a spear gun with them, they had plenty to eat. They caught lobster and fish every day, and, as one of them put it ``ate like kings''. When a passing tanker rescued them five days later, both men were genuinely sorry that they had to leave. 
\subsection{It's only me}
\label{sec-1-13}

After her husband had gone to work, Mrs Richards sent her children to school and went upstairs to her bedroom. She was too excited to do any housework that morning, for in the evening she would be going to a fancy dress party with her husband. She intended to dress up as a ghost and as she had made her costume the night before, she was impatient to try it on. Though the costume consisted only of a sheet, it was very effective. After putting it on, Mrs Richards went downstairs. She wanted to find out whether it would be comfortable to wear. 

Just as Mrs Richards was entering the dining-room, there was a knock on the front door. She knew that it must be the baker. She had told him to come straight in if ever she failed to open the door and to leave the bread on the kitchen table. Not wanting to frighten the poor man, Mrs Richards quickly hid in the small store-room under the stairs. She heard the front door open and heavy footsteps in the hall. Suddenly the door of the store-room was opened and a man entered. Mrs Richards realized that it must be the man from the Electricity Board who had come to read the meter. She tried to explain the situation, saying ``It's only me'', but it was too late. The man let out a cry and jumped back several paces. When Mrs Richards walked towards him, he fled, slamming the door behind him.
\subsection{A noble gangster}
\label{sec-1-14}

There was a time when the owners of shop and businesses in Chicago had to pay large sums of money to gangsters in return for ``protection'' If the money was not paid promptly, the gangsters would quickly put a man out of business by destroying his shop. Obtaining ``protechon money'' is not a modern crime. As long ago as the fourteenth century, an Englishman, Sir John Hawkwood, made the remarkable discovery that people would rather pay large sums of money than have their life work destroyed by gangsters.

Six hundred years ago, Sir John Hawkwood arrived in Italy with a band of soldiers and settled near Florence. He soon made a name for himself and came to be known to the Italians as Giovanni Acuto. Whenever the Italian city-states were at war with each other, Hawkwood used to hire his soldiers to princes who were willing to pay the high price he demanded. In times of peace, when business was bad, Hawkwood and his men would march into a city-state and, after burning down a few farms, would offer to go away if protection money was paid to them. Hawkwood made large sums of money in this way. In spite of this, the Italians regarded him as a sort of hero. When he died at the age of eighty, the Florentines gave him a state funeral and had a picture painted which was dedicated to the memory of ``the most valiant soldier and most notable leader, Signor Giovanni Haukodue''. 
\subsection{Fifty pence worth of trouble}
\label{sec-1-15}

Children always appreciate small gifts of money. Father, of course, provides a regular supply of pocket-money, but uncles and aunts are always a source of extra income. With some children, small sums go a long way. If sixpences are not exchanged for sweets, they rattle for months inside money-boxes. Only very thrifty children manage to fill up a money-box. For most of them, sixpence is a small price to pay for a satisfying bar of chocolate.

My nephew, George, has a money-box but it is always empty. Very few of the sixpences I have given him have found their way there. I gave him sixpence yesterday and advised him to save it. Instead, he bought himself sixpence worth of trouble. On his way to the sweet shop, he dropped his sixpence and it rolled along the pavement and then disappeared down a drain. George took off his jacket, rolled up his sleeves and pushed his right arm through the drain cover. He could not find his sixpence anywhere, and what is more, he could not get his arm out. A crowd of people gathered round him and a lady rubbed his arm with soap and butter, but George was firmly stuck. The fire-brigade was called and two firemen freed George using a special type of grease. George was not too upset by his experience because the lady who owns the sweet shop heard about his troubles and rewarded him with a large box of chocolates. 
\subsection{Mary had a little lamb}
\label{sec-1-16}

Mary and her husband Dimitri lived in the tiny village of Perachora in southern Greece. One of Mary's prize possessions was a little white lamb which her husband had given her. She kept it tied to a tree in a field during the day and went to fetch it every evening. One evening, however, the lamb was missing. The rope had been cut, so it was obvious that the lamb had been stolen. When Dimitri came in from the fields, his wife told him what had happened. Dimitri at once set out to find the thief. 

He knew it would not prove difficult in such a small village. After telling several of his friends about the theft, Dimitri found out that his neighbour, Aleko, had suddenly acquired a new lamb. Dimitri immediately went to Aleko's house and angrily accused him of stealing the lamb. He told him he had better return it or he would call the police. Aleko denied taking it and led Dimitri into his back-yard. It was true that he had just bought a lamb, he explained, but his lamb was black. Ashamed of having acted so rashly, Dimitri apologized to Aleko for having accused him. While they were talking it began to rain and Dimitri stayed in Aleko's house until the rain stopped. When he went outside half an hour later, he was astonished to find that the little black lamb was almost white. Its wool, which had been dyed black, had been washed clean by the rain!
\subsection{The longest suspension bridge in the world}
\label{sec-1-17}

Verrazano, an Italian about whom little is known, sailed into New York Harbour in 1524 and named it Angouleme. He described it as ``a very agreeable situation located within two small hills in the midst of which flowed a great river.'' Though Verrazano is by no means considered to be a great explorer, his name will probably remain immortal, for on November 21st, 1964, the greatest bridge in the world was named after him. 

The Verrazano Bridge, which was designed by Othmar Ammann, joins Brooklyn to Staten Island. It has a span of 4260 feet. The bridge is so long that the shape of the earth had to be taken into account by its designer. Two great towers support four huge cables. The towers are built on immense underwater platforms made of steel and concrete. The platforms extend to a depth of over 100 feet under the sea. These alone took sixteen months to build. Above the surface of the water, the towers rise to a height of nearly 700 feet. They support the cables from which the bridge has been suspended. Each of the four cables contains 26,108 lengths of wire. It has been estimated that if the bridge were packed with cars, it would still only be carrying a third of its total capacity. However, size and strength are not the only important things about this bridge. Despite its immensity, it is both simple and elegant, fulfilling its designer's dream to create ``an enormous object drawn as faintly as possible''. 
\subsection{Electric currents in modern art}
\label{sec-1-18}

Modern sculpture rarely surprises us any more. The idea that modern art can only be seen in museums is mistaken. Even people who take no interest in art cannot have failed to notice examples of modern sculpture on display in public places. Strange forms stand in gardens, and outside buildings and shops. We have got quite used to them. Some so-called ``modern'' pieces have been on display for nearly fifty years. 

In spite of this, some people---including myself---were surprised by a recent exhibition of modern sculpture. The first thing I saw when I entered the art gallery was a notice which said: ``Do not touch the exhibits. Some of them are dangerous!'' The objects on display were pieces of moving sculpture. Oddly shaped forms that are suspended from the ceiling and move in response to a gust of wind are quite familiar to everybody. These objects, however, were different. Lined up against the wall, there were long thin wires attached to metal spheres. The spheres had been magnetized and attracted or repelled each other all the time. In the centre of the hall, there were a number of tall structures which contained coloured lights. These lights flickered continuously like traffic lights which have gone mad. Sparks were emitted from small black boxes and red lamps flashed on and off angrily. It was rather like an exhibition of prehistoric electronic equipment. These Peculiar forms not only seemed designed to shock people emotionally, but to give them electric shocks as well! 
\subsection{A very dear cat}
\label{sec-1-19}

Kidnappers are rarely interested in Animals, but they recently took considerable interest in Mrs Eleanor Ramsay's cat. Mrs Eleanor Ramsay, a very wealthy old lady, has shared a flat with her cat, Rastus, for a great many years. Rastus leads an orderly life. He usually takes a short walk in the evenings and is always home by seven o'clock. One evening, however, he failed to arrive. Mrs Ramsay got very worried. She looked everywhere for him but could not find him.

Three day after Rastus' disappearance, Mrs Ramsay received an anonymous letter. The writer stated that Rastus was in safe hands and would be returned immediately if Mrs Ramsay paid a ransom of \&1000. Mrs Ramsay was instructed to place the money in a cardboard box and to leave it outside her door. At first, she decided to go to the police, but fearing that she would never see Rastus again---the letter had made that quite clear---she changed her mind. She drew \&1000 from her bank and followed the kidnapper's instructions. The next morning, the box had disappeared but Mrs Ramsay was sure that the kidnapper would keep his word. Sure enough, Rastus arrived punctually at seven o'clock that evening. He looked very well, though he was rather thirsty, for he drank half a bottle of milk. The police were astounded when Mrs Ramsay told them what she had done. She explained that Rastus was very dear to her. Considering the amount she paid, he was dear in more ways than one! 
\subsection{Pioneer pilots}
\label{sec-1-20}

In 1908 Lord Northcliffe offered a prize of \&1000 to the first man who would fly across the English Channel. Over a year passed before the first attempt was made. On July 19th, 1909, in the early morning, Hubert Latham took off from the French coast in his plane the ``Antoinette IV''. He had travelled only seven miles across the Channel when his engine failed and he was forced to land on the sea. The ``Antoinette'' floated on the water until Latham was picked up by a ship. 

Two days later, Louis Bleriot arrived near Calais with a plane called ``No. XI''. Bleriot had been making planes since 1905 and this was his latest model. A week before, he had completed a successful overland flight during which he covered twenty-six miles. Latham, however did not give up easily. He, too, arrived near Calais on the same day with a new ``Antonette''. It looks as if there would be an exciting race across the Channel. Both planes were going to take off on July 25th, but Latham failed to get up early enough. After making a short test flight at 4.15 a.m., Bleriot set off half an hour later. His great flight lasted thirty seven minutes. When he landed near Dover, the first person to greet him was a local policeman. Latham made another attempt a week later and got within half a mile of Dover, but he was unlucky again. His engine failed and he landed on the sea for the second time. 
\subsection{Daniel Mendoza}
\label{sec-1-21}

Boxing matches were very popular in England two hundred years ago. In those days, boxers fought with bare fists for Prize money. Because of this, they were known as ``prize-fighters''. However, boxing was very crude, for there were no rules and a prize-fighter could be seriously injured or even killed during a match. 

One of the most colourful figures in boxing history was Daniel Mendoza who was born in 1764. The use of gloves was not introduced until 1860 when the Marquis of Queensberry drew up the first set Of rules. Though he was technically a prize-fighter, Mendoza did much to change crude prize-fighting into a sport, for he brought science to the game. In his day, Mendoza enjoyed tremendous popularity. He was adored by rich and poor alike. Mendoza rose to fame swiftly after a boxing-match when he was only fourteen years old. This attracted the attention of Richard Humphries who was then the most eminent boxer in England. He offered to train Mendoza and his young pupil was quick to learn. In fact, Mendoza soon became so successful that Humphries turned against him. The two men quarrelled bitterly and it was clear that the argument could only be settled by a fight. A match was held at Stilton where both men fought for an hour. The public bet a great deal of money on Mendoza, but he was defeated. Mendoza met Humphries in the ring on a later occasion and he lost for a second time. It was not until his third match in 1790 that he finally beat Humphries and became Champion of England. Meanwhile, he founded a highly successful Academy and even Lord Byron became one of his pupils. He earned enormous sums of money and was paid as much as \&100 for a single appearance. Despite this, he was so extravagant that he was always in debt. After he was defeated by a boxer called Gentleman Jackson, he was quickly forgotten. He was sent to prison for failing to pay his debts and died in poverty in 1836. 
\subsection{By heart}
\label{sec-1-22}

Some plays are so successful that they run for years on end. In many ways, this is unfortunate for the poor actors who are required to go on repeating the same lines night after night. One would expect them to know their parts by heart and never have cause to falter. Yet this is not always the case.

A famous actor in a highly successful play was once cast in the role of an aristocrat who had been imprisoned in the Bastille for twenty years. In the last act, a gaoler would always come on to the stage with a letter which he would hand to the prisoner. Even though the noble was expected to read the letter at each performance, he always insisted that it should be written out in full. One night, the gaoler decided to play a joke on his colleague to find out if, after so many performances, he had managed to learn the contents of the letter by heart. The curtain went up on the final act of the play and revealed the aristocrat sitting alone behind bars in his dark cell. Just then, the gaoler appeared with the precious letter in his hands. He entered the cell and presented the letter to the aristocrat. But the copy he gave him had not been written out in full as usual. It was simply a blank sheet of paper. The gaoler looked on eagerly, anxious to see if his fellow-actor had at last learnt his lines. The noble stared at the blank sheet of paper for a few seconds. Then, squinting his eyes, he said: ``The light is dim. Read the letter to me.'' And he promptly handed the sheet of paper to the gaoler. Finding that he could not remember a word of the letter either, the gaoler replied: ``The light is indeed dim, sire. I must get my glasses.'' With this, he hurried off the stage. Much to the aristocrat's amusement, the gaoler returned a few moments later with a pair of glasses and the usual copy of the letter which he proceeded to read to the prisoner.
\subsection{One man's meat is another man's poison}
\label{sec-1-23}

People become quite illogical when they try to decide what can be eaten and what cannot be eaten. If you lived in the Mediterranean, for instance, you would consider octopus a great delicacy. You would not be able to understand why some people find it repulsive. On the other hand, your stomach would turn at the idea of frying potatoes in animal fat---the normally accepted practice in many northern countries. The sad truth is that most of us have been brought up to eat certain foods and we stick to them all our lives.

No creature has received more praise and abuse than the common garden snail. Cooked in wine, snails are a great luxury in various parts of the world. There are countless people who, ever since their early years, have learned to associate snails with food. My friend, Robert, lives in a country where snails are despised. As his flat is in a large town, he has no garden of his own. For years he has been asking me to collect snails from my garden and take them to him. The idea never appealed to me very much, but one day, after a heavy shower, I happened to be walking in my garden when I noticed a huge number of snails taking a stroll on some of my prize plants. Acting on a sudden impulse, I collected several dozen, put them in a paper bag, and took them to Robert. Robert was delighted to see me and equally pleased with my little gift. I left the bag in the hall and Robert and I went into the living-room where we talked for a couple of hours. I had forgotten all about the snails when Robert suddenly said that I must stay to dinner. Snails would, of course, be the main dish. I did not fancy the idea and I reluctantly followed Robert out of the room. To our dismay, we saw that there were snails everywhere: they had escaped from the paper bag and had taken complete possession of the hall! I have never been able to look at a snail since then. 
\subsection{A skeleton in the cupboard}
\label{sec-1-24}

We often read in novels how a seemingly respectable person or family has some terrible secret which has been concealed from strangers for years. The English language possesses a vivid saying to describe this sort of situation. The terrible secret is called ``a skeleton in the cup board''. At some dramatic moment in the story the terrible secret becomes known and a reputation is ruined. The reader's hair stands on end when he reads in the final pages of the novel that the heroine, a dear old lady who had always been so kind to everybody, had, in her youth, poisoned every one of her five husbands. 

It is all very well for such things to occur in fiction. To varying degrees, we all have secrets which we do not want even our closest friends to learn, but few of us have skeletons in the cupboard. The only person I know who has a skeleton in the cupboard is George Carlton, and he is very proud of the fact. George studied medicine in his youth. Instead of becoming a doctor, however, he became a successful writer of detective stories. I once spent an uncomfortable week-end which I shall never forget at his house. George showed me to the guestroom which, he said, was rarely used. He told me to unpack my things and then come down to dinner. After I had stacked my shirts and underclothes in two empty drawers, I decided to hang in the cupboard one of the two suits I had brought with me. I opened the cupboard door and then stood in front of it petrified. A skeleton was dangling before my eyes. The sudden movement of the door made it sway slightly and it gave me the impression that it was about to leap out at me. Dropping my suit, I dashed downstairs to tell George. This was worse than ``a terrible secret''; this was a real skeleton ! But George was unsympathetic. ``Oh, that,'' he said with a smile as if he were talking about an old friend. ``That's Sebastian. You forget that I was a medical student once upon a time.'' 
\subsection{The Cutty Sark}
\label{sec-1-25}

One of the most famous sailing ships of the nineteenth century, the Cutty Sark, can still be seen at Greenwich. She stands on dry land and is visited by thousands of people each year. She serves as an impressive reminder of the great ships of the past. Before they were replaced by steam-ships, sailing vessels like the Cutty Sark were used to carry tea from China and wool from Australia. The Cutty Sark was one of the fastest sailing ships that has ever been built. The only other ship to match her was the Thermopylae. Both these ships set out from Shanghai on June 18th, 1872 on an exciting race to England. This race, which went on for exactly four months, was the last of its kind. It marked the end of the great tradition of ships with sails and the beginning of a new era. 

The first of the two ships to reach Java after the race had begun was the Thermopylae, but on the Indian Ocean, the Cutty Sark took the lead. It seemed certain that she would be the first ship home, but during the race she had a lot of bad luck. In August, she was struck by a very heavy storm during which her rudder was torn away. The Cutty Sark rolled from side to side and it became impossible to steer her. A temporary rudder was made on board from spare planks and it was fitted with great difficulty. This greatly reduced the speed of the ship, for there was danger that if she travelled too quickly, this rudder would be torn away as well. Because of this, the Cutty Sark lost her lead. After crossing the equator , the captain called in at a port to have a new rudder fitted, but by now the Thermopylae was over five hundred miles ahead. Though the new rudder was fitted at tremendous speed, it was impossible for the Cutty Sark to win. She arrived in England a week after the Thermopylae. Even this was remarkable, considering that she had had so many delays. There is no doubt that if she had not lost her rudder she would have won the race easily.
\subsection{Wanted: a large biscuit tin}
\label{sec-1-26}

No one can avoid being influenced by advertisements. Much as we may pride ourselves on our good taste, we are no longer free to choose the things we want, for advertising exerts a subtle influence on us. In their efforts to persuade us to buy this or that product, advertisers have made a close study of human nature and have classified all our little weaknesses. Advertisers discovered years ago that all of us love to get something for nothing. An advertisement which begins with the magic word FREE can rarely go wrong. These days, advertisers not only offer free samples but free cars, free houses, and free trips round the world as well. They devise hundreds of competitions which will enable us to win huge sums of money. Radio and television have made it possible for advertisers to capture the attention of millions of people in this way. During a radio programme, a company of biscuit manufacturers once asked listeners to bake biscuits and send them to their factory. They offered to pay \$2 a pound for the biggest biscuit baked by a listener. The response to this competition was tremendous. Before long, biscuits of all shapes and sizes began arriving at the factory. One lady brought in a biscuit on a wheelbarrow. It weighed nearly 500 pounds. A little later, a man came along with a biscuit which occupied the whole boot of his car. All the biscuits that were sent were carefully weighed. The largest was 713 pounds. It seemed certain that this would win the prize. But just before the competition closed, a lorry arrived at the factory with a truly colossal biscuit which weighed 2400 pounds. It had been baked by a college student who had used over 1000 pounds of flour, 800 pounds of sugar, 200 pounds of fat, and 400 pounds of various other ingredients. It was so heavy that a crane had to be used to remove it from the lorry. The manufacturers had to pay more money than they had anticipated, for they bought the biscuit from the student for \$4800. 
\subsection{Nothing to sell and nothing to buy}
\label{sec-1-27}

It has been said that everyone lives by selling something. In the light of this statement, teachers live by selling knowledge, philosophers by selling wisdom and priests by selling spiritual comfort.

Though it may be possible to measure the value of material goods in terms of money, it is extremely difficult to estimate the true value of the services which people perform for us. There are times when we would willingly give everything we possess to save our lives, yet we might grudge paying a surgeon a high fee for offering us precisely this service. The conditions of society are such that skills have to be paid for in the same way that goods are paid for at a shop. Everyone has something to sell.

Tramps seem to be the only exception to this general rule. Beggars almost sell themselves as human beings to arouse the pity of passers-by. But real tramps are not beggars. They have nothing to sell and require nothing from others. In seeking independence, they do not sacrifice their human dignity. A tramp may ask you for money, but he will never ask you to feel sorry for him. He has deliberately chosen to lead the life he leads and is fully aware of the consequences He, may never be sure where the next meal is coming from, but he is free from the thousands of anxieties which afflict other people. His few material possession make it possible for him to move from place to place with ease. By having to sleep in the open, he gets far closer to the world of nature than most of us ever do. He may hunt, beg, or steal occasionally to keep himself alive; he may even in times of real need, do a little work; but he will never sacrifice his freedom. We often speak of tramps with contempt and put them in the same class as beggars, but how many of us can honestly say that we have not felt a little envious of their simple way of life and their freedom from care?
\subsection{Five pounds too dear}
\label{sec-1-28}

Small boats loaded with wares sped to the great liner as she was entering the harbour. Before she had anchored, the men from the boats had climbed on board and the decks were soon covered with colourful rugs from Persia, silks from India, copper coffee pots, and beautiful hand-made silver-ware. It was difficult not to be tempted. Many of the tourists on board had begun bargaining with the tradesmen, but I decided not to buy anything until I had disembarked. I had no sooner got off the ship than I was assailed by a man who wanted to sell me a diamond ring. I had no intention of buying one, but I could not conceal the fact that I was impressed by the size of the diamonds. Some of them were as big as marbles. The man went to great lengths to prove that the diamonds were real. As we were walking past a shop, he held a diamond firmly against the window and made a deep impression in the glass. It took me over half an hour to get rid of him.

The next man to approach me was selling expensive pens and watches. I examined one of the pens closely. It certainly looked genuine. At the base of the gold cap, the words ``made in the U.S.A.'' had been neatly inscribed. The man said that the pen was worth \&10, but as a special favour, he would let me have it for \&8. I shook my head and held up a finger indicating that I was willing to pay a pound. Gesticulating wildly, the man acted as if he found my offer outrageous, but he eventually reduced the price to \&3. Shrugging my shoulders, I began to walk away when, a moment later, he ran after me and thrust the pen into my hands. Though he kept throwing up his arms in despair, he readily accepted the pound I gave him. I felt especially pleased with my wonderful bargain---until I got back to the ship. No matter how hard I tried, it was impossible to fill this beautiful pen with ink and to this day it has never written a single word !
\subsection{Funny or not?}
\label{sec-1-29}

Whether we find a joke funny or not largely depends on where we have been brought up. The sense of humour is mysteriously bound up with national characteristics. A Frenchman, for instance, might find it hard to laugh at a Russian joke. In the same way, a Russian might fail to see anything amusing in a joke which would make an Englishman laugh to tears. 

Most funny stories are based on comic situations. In spite of national differences, certain funny situations have a universal appeal. No matter where you live, you would find it difficult not to laugh at, say, Charlie Chaplin's early films. However, a new type of humour, which stems largely from America, has recently come into fashion. It is cal1ed ``sick humour''. Comedians base their jokes on tragic situations like violent death or serious accidents. Many people find this sort of joke distasteful. The following example of ``sick humour'' will enable you to judge for yourself. 

A man who had broken his right leg was taken to hospital a few weeks before Christmas. From the moment he arrived there, he kept on pestering his doctor to tell him when he would be able to go home. He dreaded having to spend Christmas in hospital. Though the doctor did his best, the patient's recovery was slow. On Christmas day, the man still had his right leg in plaster. He spent a miserable day in bed thinking of all the fun he was missing. The following day, however, the doctor consoled him by telling him that his chances of being able to leave hospital in time for New Year celebrations were good. The man took heart and, sure enough, on New Year's Eve he was able to hobble along to a party. To compensate for his unpleasant experiences in hospital, the man drank a little more than was good for him. In the process, he enjoyed himself thoroughly and kept telling everybody how much he hated hospitals. He was still mumbling something about hospitals at the end of the party when he slipped on a piece of ice and broke his left leg. 
\subsection{The death of a ghost}
\label{sec-1-30}

For years villagers believed that Endley farm was haunted. The farm was owned by two brothers, Joe and Bert Cox. They employed a few farm hands, but no one was willing to work there long. Every time a worker gave up his job, he told the same story. Farm labourers said that they always woke up to find the work had been done overnight. Hay had been cut and cow sheds had been cleaned. A farm worker, who stayed up all night, claimed to have seen a figure cutting corn in the moonlight. In time, it became an accepted fact that the Cox brothers employed a conscientious ghost that did most of their work for them.

No one suspected that there might be someone else on the farm who had never been seen. This was indeed the case. A short time ago, villagers were astonished to learn that the ghost of Endley had died. Everyone went to the funeral, for the ``ghost'' was none other than Eric Cox, a third brother who was supposed to have died as a young man. After the funeral, Joe and Bert revealed a secret which they had kept for over forty years. Eric had been the eldest son of the family. He had been obliged to join the army during the first World War. As he hated army life he decided to desert his regiment. When he learnt that he would be sent abroad, he returned to the farm and his farther hid him until the end of the war. Fearing the authorities, Eric remained in hiding after the war as well. His father told everybody that Eric had been killed in action. The only other people who knew the secret were Joe and Bert. They did not even tell their wives. When their father died, they thought it their duty to keep Eric in hiding. All these years, Eric had lived as a recluse(隐遁者, 寂寞者). He used to sleep during the day and work at night, quite unaware of the fact that he had become the ghost of Endley. When he died, however, his brothers found it impossible to keep the secret any longer. 
\subsection{A lovable eccentric}
\label{sec-1-31}

True eccentrics never deliberately set out to draw attention to themselves. They disregard social conventions without being conscious that they are doing anything extraordinary. This invariably wins them the love and respect of others, for they add colour to the dull routine of everyday life. 

Up to the time of his death, Richard Colson was one of the most notable figures in our town. He was a shrewd and wealthy business-man, but the ordinary town-folk hardly knew anything about this side of his life. He was known to us all as Dickie and his eccentricity had become legendary long before he died. Dickie disliked snobs (势利小人) intensely. Though he owned a large car, he hardly ever used it, preferring always to go on foot. Even when it was raining heavily, he refused to carry an umbrella. One day, he walked into an expensive shop after having been caught in a particularly heavy shower. He wanted to buy a \&300 fur coat for his wife, but he was in such a bedraggled condition that an assistant refused to serve him. Dickie left the shop without a word and returned carrying a large cloth bag. As it was extremely heavy, he dumped it on the counter. The assistant asked him to leave, but Dickie paid no attention to him and requested to see the manager. Recognizing who the customer was, the manager was most apologetic and reprimanded the assistant severely. When Dickie was given the fur coat, he presented the assistant with the cloth bag. It contained \&300 in pennies. He insisted on the assistant's counting the money before he left 72,000 pennies in all! On another occasion, he invited a number of important critics to see his private collection of modern paintings. This exhibition received a great deal of attention in the press, for though the pictures were supposed to be the work of famous artists, they had in fact been painted by Dickie. It took him four years to stage this elaborate joke simply to prove that critics do not always know what they are talking about.
\subsection{A lost ship}
\label{sec-1-32}

The salvage operation had been a complete failure. The small ship, Elkor, which had been searching the Barents Sea for weeks, was on its way home. A radio message from the mainland had been received by the ship's captain instructing him to give up the search. The captain knew that another attempt would be made later, for the sunken ship he was trying to find had been carrying a precious cargo of gold bullion. 

Despite the message, the captain of the Elkor decided to try once more. The sea-bed was scoured with powerful nets and there was tremendous excitement on board when a chest was raised from the bottom. Though the crew were at first under the impression that the lost ship had been found, the contents of the sea-chest proved them wrong. What they had in fact found was a ship which had been sunk many years before. The chest contained the personal belongings of a seaman, Alan Fielding. There were books, clothing and photographs, together with letters which the seaman had once received from his wife. The captain of the Elkor ordered his men to salvage as much as possible from the wreck. Nothing of value was found, but the numerous items which were brought to the surface proved to be of great interest. From a heavy gun that was raised, the captain realized that the ship must have been a cruiser. In another sea-chest, which contained the belongings of a ship's officer, there was an unfinished letter which had been written on March 14th, 1943. The captain learnt from the letter that the name of the lost ship was the Karen. The most valuable find of all was the ship's log book, parts of which it was still possible to read. From this the captain was able to piece together all the information that had come to light. The Karen had been sailing in a convoy to Russia when she was torpedoed by an enemy submarine. This was later confirmed by a naval official at the Ministry of Defence after the Elkor had returned home. All the items that were found were sent to the War Museum.
\subsection{A day to remember}
\label{sec-1-33}

We have all experienced days when everything goes wrong. A day may begin well enough, but suddenly everything seems to get out of control. What invariably happens is that a great number of things choose to go wrong at precisely the same moment. It is as if a single unimportant event set up a chain of reactions. Let us suppose that you are preparing a meal and keeping an eye on the baby at the same time. The telephone rings and this marks the prelude to an unforeseen series of catastrophes. While you are on the phone, the baby pulls the table-cloth off the table smashing half your best crockery and cutting himself in the process. You hang up hurriedly and attend to baby, crockery, etc. Meanwhile, the meal gets burnt. As if this were not enough to reduce you to tears, your husband arrives, unexpectedly bringing three guests to dinner.

Things can go wrong on a big scale as a number of people recently discovered in Parramatta, a suburb of Sydney. During the rush hour one evening two cars collided and both drivers began to argue. The woman immediately behind the two cars happened to be a learner. She suddenly got into a panic and stopped her car. This made the driver following her brake hard. His wife was sitting beside him holding a large cake. As she was thrown forward, the cake went right through the windscreen and landed on the road. Seeing a cake flying through the air, a lorry-driver who was drawing up alongside the car, pulled up all of a sudden. The lorry was loaded with empty beer bottles and hundreds of them slid off the back of the vehicle and on to the road. This led to yet another angry argument. Meanwhile, the traffic piled up behind. It took the police nearly an hour to get the traffic on the move again. In the meantime, the lorry-driver had to sweep up hundreds of broken bottles. Only two stray dogs benefited from all this confusion, for they greedily devoured what was left of the cake. It was just one of those days!
\subsection{A happy discovery}
\label{sec-1-34}

Antique shops exert a peculiar fascination on a great many people. The more expensive kind of antique shop where rare objects are beautifully displayed in glass cases to keep them free from dust is usually a forbidding place. But no one has to muster up courage to enter a less pretentious antique shop. There is always hope that in its labyrinth of musty, dark, disordered rooms a real rarity will be found amongst the piles of assorted junk that litter the floors.

No one discovers a rarity by chance. A truly dedicated searcher for art treasures must have patience, and above all, the ability to recognize the worth of something when he sees it. To do this, he must be at least as knowledgeable as the dealer. Like a scientist bent on making a discovery, he must cherish the hope that one day he will be amply rewarded. 

My old friend, Frank Halliday, is just such a person. He has often described to me how he picked up a masterpiece for a mere \&5. One Saturday morning, Frank visited an antique shop in my neighbourhood. As he had never been there before, he found a great deal to interest him. The morning passed rapidly and Frank was about to leave when he noticed a large packing-case lying on the floor. The dealer told him that it had just come in, but that he could not be bothered to open it. Frank begged him to do so and the dealer reluctantly prised it open. The contents were disappointing. Apart from an interesting-looking carved dagger, the box was full of crockery, much of it broken. Frank gently lifted the crockery out of the box and suddenly noticed a miniature Painting at the bottom of the packing-case. As its composition and line reminded him of an Italian painting he knew well, he decided to buy it. Glancing at it briefly, the dealer told him that it was worth \&5. Frank could hardly conceal his excitement, for he knew that he had made a real discovery. The tiny painting proved to be an unknown masterpiece by Correggio and was worth thousands of pounds. 
\subsection{Justice was done}
\label{sec-1-35}

The word justice is usually associated with courts of law. We might say that justice has been done when a man's innocence or guilt has been proved beyond doubt. Justice is part of the complex machinery of the law. Those who seek it, undertake an arduous journey and can never be sure that they will find it. Judges, however wise or eminent, are human and can make mistakes. 

There are rare instances when justice almost ceases to be an abstract conception. Reward or punishment are out quite independent of human interference. At such times, justice acts like a living force. When we use a phrase like it serves him right, we are, in part, admitting that a certain set of circumstances has enabled justice to act of its own accord.

When a thief was caught on the premises of a large fur store one morning, the shop assistants must have found it impossible to resist the temptation to say ``it serves him right''. The shop was an old-fashioned one with many large, disused fireplaces and tall, narrow chimneys. Towards midday, a girl heard a muffled cry coming from behind one of the walls. As the cry was repeated several times, she ran to tell the manager who promptly rang up the fire-brigade. The cry had certainly come from one of the chimneys, but as there were so many of them, the firemen could not be certain which one it was. They located the right chimney by tapping at the walls and listening for the man's cries. After chipping through a wall which was eighteen inches thick, they found that a man had been trapped in the chimney. As it was extremely narrow, the man was unable to move, but the firemen were eventually able to free him by cutting a huge hole in the wall. The sorry-looking, blackened figure that emerged, at once admitted that he had tried to break into the shop during the night but had got stuck in the chimney. He had been there for nearly ten hours. Justice had been done even before the man was handed over to the police. 
\subsection{A chance in a million}
\label{sec-1-36}

We are less credulous than we used to be In the nineteenth century, a novelist would bring his story to a conclusion by presenting his readers with a series of coincidences---most of them wildly improbable. Readers happily accepted the fact that an obscure maid-servant was really the hero's mother. A long-lost brother, who was presumed dead, was really alive all the time and wickedly plotting to bring about the hero's down-fall. And so on. Modern readers would find such naive solutions totally unacceptable. Yet, in real life, circumstances do sometimes conspire to bring about coincidences which anyone but a nineteenth century novelist would find incredible.

A German taxi-driver, Franz Bussman, recently found a brother who was thought to have been killed twenty years before. While on a walking tour with his wife, he stopped to talk to a workman. After they had gone on, Mrs Bussman commented on the workman's close resemblance to her husband and even suggested that he might be his brother. Franz poured scorn on the idea, pointing out that his brother had been killed in action during the war. Though Mrs Bussman was fully acquainted with this story, she thought that there was a chance in a million that she might be right. A few days later, she sent a boy to the workman to ask him if his name was Hans Bussman, Needless to say, the man's name was Hans Bussman and he really was Franz's long-lost brother.

When the brothers were re-united, Hans explained how it was that he was still alive. After having been wounded towards the end of the war, he had been sent to hospital and was separated from his unit. The hospital had been bombed and Hans had made his way back into Western Germany on foot. Meanwhile, his unit was lost and all records of him had been destroyed. Hans returned to his family home, but the house had been bombed and no one in the neighbourhood knew what had become of the inhabitants. Assuming that his family had been killed during an air-raid, Hans settled down in a Village fifty miles away where he had remained ever since. 
\subsection{The Westhaven Express}
\label{sec-1-37}

We have learnt to expect that trains will be punctual. After years of pre-conditioning, most of us have developed an unshakable faith in railway time-tables. Ships may be delayed by storms; air flights may be cancelled because of bad weather; but trains must be on time. Only an exceptionally heavy snow fall might temporarily dislocate railway services. It is all too easy to blame the railway authorities when something does go wrong. The truth is that when mistakes occur, they are more likely to be ours than theirs.

After consulting my railway time-table, I noted with satisfaction that there was an express train to Westhaven. It went direct from my local station and the journey lasted a mere hour and seventeen minutes. When I boarded the train, I could not help noticing that a great many local people got on as well. At the time, this did not strike me as odd. I reflected that there must be a great many people besides myself who wished to take advantage of this excellent service. Neither was I surprised when the train stopped at Widley, a tiny station a few miles along the line. Even a mighty express train can be held up by signals. But when the train dawdled at station after station, I began to wonder. It suddenly dawned on me that this express was not roaring down the line at ninety miles an hour, but barely chugging along at thirty. One hour and seventeen minutes passed and we had not even covered half the distance. I asked a passenger if this was the Westhaven Express, but he had not even heard of it. I determined to lodge a complaint as soon as we arrived. Two hours later, I was talking angrily to the station-master at Westhaven. When he denied the train's existence, I borrowed his copy of the time-table. There was a note of triumph in my voice when I told him that it was there in black and white. Glancing at it briefly, he told me to look again. A tiny asterisk conducted me to a footnote at the bottom of the page. It said: ``This service has been suspended.'' 
\subsection{The first calendar}
\label{sec-1-38}

Future historians will be in a unique position when they come to record the history of our own times. They will hardly know which facts to select from the great mass of evidence that steadily accumulates. What is more they will not have to rely solely on the written word. Films, gramophone records, and magnetic tapes will provide them with a bewildering amount of information. They will be able, as it were, to see and hear us in action. But the historian attempting to reconstruct the distant past is always faced with a difficult task. He has to deduce what he can from the few scanty clues available. Even seemingly insignificant remains can shed interesting light on the history of early man. 

Up to now, historians have assumed that calendars came into being with the advent of agriculture, for then man was faced with a real need to understand something about the seasons. Recent scientific evidence seems to indicate that this assumption is incorrect. Historians have long been puzzled by dots, lines and symbols which have been engraved on walls, bones, and the ivory tusk of mammoths. The nomads who made these markings lived by hunting and fishing during the last Ice Age, which began about 35,000 B.C. and ended about 10,000 B.C. By correlating markings made in various parts of the world, historians have been able to read this difficult code. They have found that it is connected with the passage of days and the phases of the moon. It is, in fact, a, primitive type of calendar. It has long been known that the hunting scenes depicted on walls were not simply a form of artistic expression. They had a definite meaning, for they were as near as early man could get to writing. It is possible that there is a definite relation between these paintings and the markings that sometimes accompany them. It seems that man was making a real effort to understand the seasons 20,000 years earlier than has been supposed.
\subsection{Nothing to worry about}
\label{sec-1-39}

The rough road across the plain soon became so bad that we tried to get Bruce to drive back to the village we had come from. Even though the road was littered with boulders and pitted with holes, Bruce was not in the least perturbed. Glancing at his map, he informed us that the next village was a mere twenty miles away. It was not that Bruce always underestimated difficulties. He simply had no sense of danger at all. No matter what the conditions were, he believed that a car should be driven as fast as it could possibly go.

As we bumped over the dusty track, we swerved to avoid large boulders. The wheels scooped up stones which hammered ominously under the car. We felt sure that sooner or later a stone would rip a hole in our petrol tank or damage the engine. Because of this, we kept looking back, wondering if we were leaving a trail of oil and petrol behind us. What a relief it was when the boulders suddenly disappeared, giving way to a stretch of plain where the only obstacles were clumps of bushes. But there was worse to come. Just ahead of us there was a huge fissure. In response to renewed pleadings, Bruce stopped. Though we all got out to examine the fissure, he remained in the car. We informed him that the fissure extended for fifty yards and was two feet wide and four feet deep. Even this had no effect. Bruce engaged low gear and drove at a terrifying speed, keeping the front wheels astride the crack as he followed its zig-zag course. Before we had time to worry about what might happen, we were back on the plain again. Bruce consulted the map once more and told us that the village was now only fifteen miles away. Our next obstacle was a shallow pool of water about half a mile across. Bruce charged at it, but in the middle, the car came to a grinding halt. A yellow light on the dash-board flashed angrily and Bruce cheerfully announced that there was no oil in the engine!
\subsection{Who's who}
\label{sec-1-40}

It has never been explained why university students seem to enjoy practical jokes more than anyone else. Students specialize in a particular type of practical joke: the hoax. Inviting the fire-brigade to put out a non-existent fire is a crude form of deception which no self-respecting student would ever indulge in, Students often create amusing situations which are funny to everyone except the victims. When a student recently saw two workmen using a pneumatic drill outside his university, he immediately telephoned the police and informed them that two students dressed up as workmen were tearing up the road with a pneumatic drill. As soon as he had hung up, he went over to the workmen and told them that if a policeman ordered them to go away, they were not to take him seriously. He added that a student had dressed up as a policeman and was playing all sorts of silly jokes on people. Both the police and the workmen were grateful to the student for this piece of advance information. 

The student hid in an archway nearby where he could watch and hear everything that went on. Sure enough, a policeman arrived on the scene and politely asked the workmen to go away. When he received a very rude reply from one of the workmen, he threatened to remove them by force. The workmen told him to do as he pleased and the policeman telephoned for help. Shortly afterwards, four more policemen arrived and remonstrated with the workmen. As the men refused to stop working, the police attempted to seize the pneumatic drill. The workmen struggled fiercely and one of them lost his temper. He threatened to call the police. At this, the police pointed out ironically that this would hardly be necessary as the men were already under arrest. Pretending to speak seriously, one of the workmen asked if he might make a telephone call before being taken to the station. Permission was granted and a policeman accompanied him to a call-box. Only when he saw that the man was actually telephoning the police did he realize that they had all been the victims of a hoax. 
\subsection{Illusions of Pastoral peace}
\label{sec-1-41}

The quiet life of the country has never appealed to me. City born and city bred, I have always regarded the country as something you look at through a train window, or something you occasionally visit during the week-end. Most of my friends live in the city, yet they always go into raptures at the mere mention of the country. Though they extol the virtues of the peaceful life, only one of them has ever gone to live in the country and he was back in town within six months. Even he still lives under the illusion that country life is somehow superior to town life. He is forever talking about the friendly people, the clean atmosphere, the closeness to nature and the gentle pace of living. Nothing can be compared, he maintains, with the first cock crow, the twittering of birds at dawn, the sight of the rising sun glinting on the trees and pastures. This idyllic pastoral scene is only part of the picture. My friend fails to mention the long and friendless winter evenings which are interrupted only by an occasional visit to the local cinema-virtually the only form of entertainment. He says nothing about the poor selection of goods in the shops, or about those unfortunate people who have to travel from the country to the city every day to get to work. Why people are prepared to tolerate a four hour journey each day for the dubious privilege of living in the country is beyond my ken. They could be saved so much misery and expense if they chose to live in the city where they rightly belong. 

If you can do without the few pastoral pleasures of the country, you will find the city can provide you with the best that life can offer. You never have to travel miles to see your friends. They invariably live nearby and are always available for an informal chat or an evening's entertainment. Some of my acquaintances in the country come up to town once or twice a year to visit the theatre as a special treat. For them this is a major operation which involves considerable planning. As the play draws to its close, they wonder whether they will ever catch that last train home. The city dweller never experiences anxieties of this sort. The latest exhibitions, films, or plays are only a short bus ride away. Shopping, too, is always a pleasure. There is so much variety that you never have to make do with second best. Country people run wild when they go shopping in the city and stagger home loaded with as many of the necessities of life as they can carry. Nor is the city without its moments of beauty. There is something comforting about the warm glow shed by advertisements on cold wet winter nights. Few things could be more impressive than the peace that descends on deserted city streets at week-ends when the thousands that travel to work every day are tucked a way in their homes in the country. It has always been a mystery to me why city dwellers, who appreciate all these things, obstinately pretend that they would prefer to live in the country. 
\subsection{Modern Cavemen}
\label{sec-1-42}

Cave exploration, or potholing, as it has come to be known, is a relatively new sport. Perhaps it is the desire for solitude or the chance of making an unexpected discovery that lures men down to the depths of the earth. It is impossible to give a satisfactory explanation for a pot-holer's motives. For him, caves have the same peculiar fascination which high mountains have for the climber. They arouse instincts which can only be dimly understood. Exploring really deep caves is not a task for the Sunday afternoon rambler. Such undertakings require the precise planning and foresight of military operations. It can take as long as eight days to rig up rope ladders and to establish supply bases before a descent can be made into a very deep cave. Precautions of this sort are necessary, for it is impossible to foretell the exact nature of the difficulties which will confront the potholer. The deepest known cave in the world is the Gouffre Berger near Grenoble. It extends to a depth of 3723 feet. This immense chasm has been formed by an underground stream which has tunnelled a course through a flaw in the rocks. The entrance to the cave is on a plateau in the Dauphine Alps. As it is only six feet across, it is barely noticeable. The cave might never have been discovered had not the entrance been spotted by the distinguished French potholer, Berger. Since its discovery, it has become a sort of potholers' Everest. Though a number of descents have been made, much of it still remains to be explored. 

A team of potholers recently went down the Gouffre Berger. After entering the narrow gap on the plateau, they climbed down the steep sides of the cave until they came to a narrow corridor. They had to edge their way along this, sometimes wading across shallow streams, or swimming across deep pools. Suddenly they came to a waterfall which dropped into an underground lake at the bottom of the cave. They plunged into the lake, and after loading their gear on an inflatable rubber dinghy, let the current carry them to the other side. To protect themselves from the icy water, they had to wear special rubber suits. At the far end of the lake, they came to huge piles of rubble which had been washed up by the water. In this part of the cave, they could hear an insistent booming sound which they found was caused by a small water-spout shooting down into a pool from the roof of the cave. Squeezing through a cleft in the rocks, the potholers arrived at an enormous cavern, the size of a huge concert hall. After switching on powerful arc lights, they saw great stalagmites-some of them over forty feet high---rising up like tree-trunks to meet the stalactites suspended from the roof. Round about, piles of lime-stone glistened in all the colours of the rainbow. In the eerie silence of the cavern, the only sound that could be heard was made by water which dripped continuously from the high dome above them.
\subsection{Fully insured}
\label{sec-1-43}

Insurance companies are normally willing to insure anything. Insuring public or private property is a standard practice in most countries in the world. If, however, you were holding an open air garden party or a fete it would be equally possible to insure yourself in the event of bad weather. Needless to say, the bigger the risk an insurance company takes, the higher the premium you will have to pay. It is not uncommon to hear that a ship-ping company has made a claim for the cost of salvaging a sunken ship. But the claim made by a local authority to recover the cost of salvaging a sunken pie dish must surely be unique. 

Admittedly it was an unusual pie dish, for it was eighteen feet long and six feet wide. It had been purchased by a local authority so that an enormous pie could be baked for an annual fair. The pie committee decided that the best way to transport the dish would be by canal, so they insured it for the trip. Shortly after it was launched, the pie committee went to a local inn to celebrate. At the same time, a number of teenagers climbed on to the dish and held a little party of their own. Modern dances proved to be more than the disk could bear, for during the party it capsized and sank in seven feet of water. 

The pie committee telephoned a local garage owner who arrived in a recovery truck to salvage the pie dish. Shivering in their wet clothes, the teenagers looked on while three men dived repeatedly into the water to locate the dish. They had little difficulty in finding it, but hauling it out of the water proved to be a serious problem. The sides of the dish were so smooth that it was almost impossible to attach hawsers and chains to the rim without damaging it. Eventually chains were fixed to one end of the dish and a powerful winch was put into operation. The dish rose to the surface and was gently drawn towards the canal bank. For one agonizing moment, the dish was perched precariously on the bank of the canal, but it suddenly overbalanced and slid back into the water. The men were now obliged to try once more. This time they fixed heavy metal clamps to both sides of the dish so that they could fasten the chains. The dish now had to be lifted vertically because one edge was resting against the side of the canal. The winch was again put into operation and one of the men started up the truck. Several minutes later, the dish was successfully hauled above the surface of the water. Water streamed in torrents over its sides with such force that it set up a huge wave in the canal. There was danger that the wave would rebound off the other side of the bank and send the dish plunging into the water again. By working at tremendous speed, the men managed to get the dish on to dry land before the wave returned. 
\subsection{Speed and comfort}
\label{sec-1-44}

People travelling long distances frequently have to decide whether they would prefer to go by land, sea, or air. Hardly anyone can positively enjoy sitting in a train for more than a few hours. Train compartments soon get cramped and stuffy. It is almost impossible to take your mind off the journey. Reading is only a partial solution, for the monotonous rhythm of the wheels clicking on the rails soon lulls you to sleep. During the day, sleep comes in snatches. At night, when you really wish to go to sleep, you rarely manage to do so. If you are lucky enough to get a couchette, you spend half the night staring at the small blue light in the ceiling, or fumbling to find your passport when you cross a frontier. Inevitably you arrive at your destination almost exhausted. Long car journeys are even less pleasant, for it is quite impossible even to read. On motor-ways you can, at least, travel fairly safely at high speeds, but more often than not, the greater part of the journey is spent on narrow, bumpy roads which are crowded with traffic. By comparison, trips by sea offer a great variety of civilized comforts. You can stretch your legs on the spacious decks, play games, swim, meet interesting people and enjoy good food---always assuming, of course, that the sea is calm. If it is not, and you are likely to get sea-sick, no form of transport could be worse. Even if you travel in ideal weather, sea journeys take a long time. Relatively few people are prepared to sacrifice up to a third of their holidays for the pleasure of travelling on a ship. 

Aeroplanes have the reputation of being dangerous and even hardened travellers are intimidated by them. They also have the grave disadvantage of being the most expensive form of transport. But nothing can match them for speed and comfort. Travelling at a height of 30,000 feet, far above the clouds, and at over 500 miles an hour is an exhilarating experience. You do not have to devise ways of taking your mind off the journey, for an aeroplane gets you to your destination rapidly. For a few hours, you settle back in a deep armchair to enjoy the flight. The real escapist can watch a free film show and sip champagne on some services. But even when such refinements are not available, there is plenty to keep you occupied. An aeroplane offers you an unusual and breathtaking view of the world. You soar effortlessly over high mountains and deep valleys. You really see the shape of the land. If the landscape is hidden from view, you can enjoy the extraordinary sight of unbroken cloud plains that stretch out for miles before you, while the sun shines brilliantly in a clear sky. The journey is so smooth that there is nothing to prevent you from reading or sleeping. However you decide to spend your time, one thing is certain: you will arrive at your destination fresh and uncrumpled. You will not have to spend the next few days recovering from a long and arduous journey. 
\subsection{The power of press}
\label{sec-1-45}

In democratic countries any efforts to restrict the freedom of the press are rightly condemned. However, this freedom can easily be abused. Stories about people often attract far more public attention than political events. Though we may enjoy reading about the lives of others, it is extremely doubtful whether we would equally enjoy reading about ourselves. Acting on the contention that facts are sacred, reporters can cause untold suffering to individuals by publishing details about their private lives. Newspapers exert such tremendous influence that they can not only bring about major changes to the lives of ordinary people but can even overthrow a government. 

The story of a poor family that acquired fame and fortune overnight, dramatically illustrates the power of the press. The family lived in Aberdeen, a small town of 23,000 inhabitants in South Dakota. As the parents had five children, life was a perpetual struggle against poverty. They were expecting their sixth child and faced with even more pressing economic problems. If they had only had one more child, the fact would have passed unnoticed. They would have continued to struggle against economic odds and would have lived in obscurity. But they suddenly became the parents of quintuplets, four girls and a boy, an event which radically changed their lives. The day after the birth of the five children, an aeroplane arrived in Aberdeen bringing sixty reporters and photographers. The news was of national importance, for the poor couple had become the parents of the only quintuplets in America. 

The rise to fame was swift. Television cameras and newspapers carried the news to everyone in the country. Newspapers and magazines offered the family huge sums for the exclusive rights to publish stories and photographs. Gifts poured in not only from unknown people, but from baby food and soap manufacturers who wished to advertise their products. The old farmhouse the family lived in was to be replaced by a new \$100,000 home. Reporters kept pressing for interviews so lawyers had to be employed to act as spokesmen for the family at press conferences. The event brought serious changes to the town itself. Plans were announced to build a huge new highway, as Aberdeen was now likely to attract thousands of tourists. Signposts erected on the outskirts of the town directed tourists not to Aberdeen, but to ``Quint-City U.S.A.'' The local authorities discussed the possibility of erecting a ``quint museum'' to satisfy the curiosity of the public and to protect the family from inquisitive tourists. While the five babies were still quietly sleeping in oxygen tents in a hospital nursery, their parents were paying the price for fame. It would never again be possible for them to lead normal lives. They had become the victims of commercialization, for their names had acquired a market value. The town itself received so much attention that almost every one of the inhabitants was affected to a greater or less degree.
\subsection{Do it yourself}
\label{sec-1-46}

So great is our passion for doing things for ourselves, that we are becoming increasingly less dependent on specialized labour. No one can plead ignorance of a subject any longer, for there are countless do-it-yourself publications. Armed with the right tools and materials, newly-weds gaily embark on the task of decorating their own homes. Men of all ages spend hours of their leisure time installing their own fireplaces, laying-out their own gardens; building garages and making furniture. Some really keen enthusiasts go so far as to build their own record players and radio transmitters. Shops cater for the do-it-yourself craze not only by running special advisory services for novices, but by offering consumers bits and pieces which they can assemble at home. Such things provide an excellent outlet for pent-up creative energy, but unfortunately not all of us are born handymen. 

Wives tend to believe that their husbands are infinitely resourceful and versatile. Even husbands who can hardly drive a nail in straight are supposed to be born electricians, carpenters, plumbers and mechanics. When lights fuse, furniture gets rickety, pipes get clogged, or vacuum cleaners fail to operate, wives automatically assume that their husbands will somehow put things right. 

The worst thing about the do-it-yourself game is that sometimes husbands live under the delusion that they can do anything even when they have been repeatedly proved wrong. It is a question of pride as much as anything else. Last spring my wife suggested that I call in a man to look at our lawn-mower. It had broken down the previous summer, and though I promised to repair it, I had never got round to it. I would not hear of the suggestion and said that I would fix it myself. One Saturday afternoon, I hauled the machine into the garden and had a close look at it. As far as I could see, it only needed a minor adjustment: a turn of a screw here, a little tightening up there, a drop of oil and it would be as good as new. Inevitably the repair job was not quite so simple. The mower firmly refused to mow, so I decided to dismantle it. The garden was soon littered with chunks of metal which had once made up a lawn-mower. But I was extremely pleased with myself I had traced the cause of the trouble. One of the links in the chain that drives the wheels had snapped. After buying a new chain I was faced with the insurmountable task of putting the confusing jigsaw puzzle together again. I was not surprised to find that the machine still refused to work after I had reassembled it, for the simple reason that I was left with several curiously shaped bits of metal which did not seem to fit anywhere. I gave up in despair. The weeks passed and the grass grew. When my wife nagged me to do something about it, I told her that either I would have to buy a new mower or let the grass grow. Needless to say our house is now surrounded by a jungle. Buried somewhere in deep grass there is a rusting lawn-mower which I have promised to repair one day.
\subsection{Through the earth's crust}
\label{sec-1-47}

Satellites orbiting round the earth have provided scientists with a vast amount of information about conditions in outer space. By comparison, relatively little is known about the internal structure of the earth. It has proved easier to go up than to go down. The deepest hole ever to be bored on land went down 25,340 feet---considerably less than the height of Mount Everest. Drilling a hole under the sea has proved to be even more difficult. The deepest hole bored under sea has been about 20,000 feet. Until recently, scientists have been unable to devise a drill which would be capable of cutting through hard rock at great depths. This problem has now been solved. Scientists have developed a method which sounds surprisingly simple. A new drill which is being tested at Leona Valley Ranch in Texas is driven by a turbine engine which is propelled by liquid mud pumped into it from the surface. As the diamond tip of the drill revolves, it is lubricated by mud. Scientists have been amazed to find that it can cut through the hardest rock with great ease. The drill has been designed to bore through the earth to a depth of 35,000 feet. It will enable scientists to obtain samples of the mysterious layer which lies immediately below the earth's crust. This layer is known as the Mohorovicic Discontinuity, but is commonly referred to as ``the Moho''.

Before it is possible to drill this deep hole, scientists will have to overcome a number of problems. Geological tests will be carried out to find the point at which the earth's crust is thinnest. The three possible sites which are being considered are all at sea: two in the Atlantic Ocean and one in the Pacific. Once they have determinded on a site, they will have to erect a drilling vessel which will not be swept away by ocean currents. The vessel will consist of an immense platform which will rise to 70 feet above the water. It will be supported by six hollow columns which will descend to a depth of 60 feet below the ocean surface where they will be fixed to a huge float. A tall steel tower rising to a height of nearly 200 feet will rest on the platform. The drill will be stored in the tower and will have to be lowered through about 15,000 feet of water before operations can begin. Within the tower, there will be a laboratory, living accommodation and a helicopter landing station. Keeping the platform in position at sea will give rise to further problems. To do this, scientists will have to devise methods using radar and underwater television. If, during the operations the drill has to be withdrawn, it must be possible to re-insert it. Great care will therefore have to be taken to keep the platform steady and make it strong enough to withstand hurricanes. If the project is successful, scientists will not only learn a great deal about the earth, but possibly about the nature of the universe itself. 
\subsection{The silent village}
\label{sec-1-48}

In this much-travelled world, there are still thousands of places which are inaccessible to tourists. We always assume that villagers in remote places are friendly and hospitable. But people who are cut off not only from foreign tourists, but even from their own countrymen can be hostile to travellers. Visits to really remote villages are seldom enjoyable---as my wife and I discovered during a tour through the Balkans. 

We had spent several days in a small town and visited a number of old churches in the vicinity. These attracted many visitors for they were not only of great architectural interest, but contained a large number of beautifully preserved frescoes as well. On the day before our departure, several bus loads of tourists descended on the town. This was more than we could bear, so we decided to spend our last day exploring the countryside. Taking a path which led out of the town, we crossed a few fields until we came to a dense wood. We expected the path to end abruptly, but we found that it traced its way through the trees. We tramped through the wood for over two hours until we arrived at a deep stream. We could see that the path continued on the other side, but we had no idea how we could get across the stream. Suddenly my wife spotted a boat moored to the bank. In it there was a boatman fast asleep. We gently woke him up and asked him to ferry us to the other side. Though he was reluctant to do so at first, we eventually persuaded him to take us. The path led to a tiny village perched on the steep sides of a mountain. The place consisted of a straggling unmade road which was lined on either side by small houses. Even under a clear blue sky, the village looked forbidding, as all the houses were built of grey mud bricks. The village seemed deserted, the only sign of life being an ugly-looking black goat tied to a tree on a short length of rope in a field nearby. Sitting down on a dilapidated wooden fence near the field, we opened a couple of tins of sardines and had a picnic lunch. All at once, I noticed that my wife seemed to be filled with alarm. Looking up I saw that we were surrounded by children in rags who were looking at us silently as we ate. We offered them food and spoke to them kindly, but they remained motionless. I concluded that they were simply shy of strangers. When we later walked down the main street of the village, we were followed by a silent procession of children. The village which had seemed deserted, immediately came to life. Faces appeared at windows. Men in shirt sleeves stood outside their houses and glared at us. Old women in black shawls peered at us from door-ways. The most frightening thing of all was that not a sound could be heard. There was no doubt that we were unwelcome visitors. We needed no further warning. Turning back down the main street, we quickened our pace and made our way rapidly towards the stream where we hoped the boatman was waiting. 
\subsection{The Ideal Servant}
\label{sec-1-49}

It is a good thing my aunt Harriet died years ago. If she were alive today she would not be able to air her views on her favourite topic of conversation: domestic servants. Aunt Harriet lived in that leisurely age when servants were employed to do housework. She had a huge, rambling country house called ``The Gables''. She was sentimentally attached to this house, for even though it was far too big for her needs, she persisted in living there long after her husband's death. Before she grew old, aunt Harriet used to entertain lavishly. I often visited The Gables when I was a boy. No matter how many guests were present, the great house was always immaculate. The parquet floors shone like mirrors; highly polished silver was displayed in gleaming glass cabinets; even my uncle's huge collection of books was kept miraculously free from dust. Aunt Harriet presided over an invisible army of servants that continuously scrubbed, cleaned, and polished. She always referred to them as ``the shifting population'', for they came and went with such frequency that I never even got a chance to learn their names, Though my aunt pursued what was, in those days, an enlightened policy in that she never allowed her domestic staff to work more than eight hours a day, she was extremely difficult to please. While she always decried the fickleness of human nature, she carried on an unrelenting search for the ideal servant to the end of her days, even after she had been sadly disillusioned by Bessie. Bessie worked for aunt Harriet for three years. During that time she so gained my aunt's confidence, that she was put in charge of the domestic staff.

Aunt Hariet could not find words to praise Bessie's industry and efficiency. In addition to all her other qualifications, Bessie was an expert cook. She acted the role of the perfect servant for three years before aunt Harriet discovered her ``little weakness''. After being absent from The Gables for a week, my aunt unexpectedly returned one afternoon with a party of guests and instructed Bessie to prepare dinner. Not only was the meal well below the usual standard, but Bessie seemed unable to walk steadily. She bumped into the furniture and kept mumbling about the guests. When she came in with the last course—a huge pudding-she tripped on the carpet and the pudding went flying through the air, narrowly missed my aunt, and crashed on the dining table with considerable force. Though this occasioned great mirth among the guests, aunt Harriet was horrified. She reluctantly came to the conclusion that Bessie was drunk. The guests had, of course, realized this from the moment Bessie opened the door for them and, long before the final catastrophe, had had a difficult time trying to conceal their amusement. The poor girl was dismissed instantly. After her departure, aunt Harriet discovered that there were piles of empty wine bottles of all shapes and sizes neatly stacked in what had once been Bessie's wardrobe. They had mysteriously found their way there from the wine-cellar! 
\subsection{New Year Resolutions}
\label{sec-1-50}

The New Year is a time for resolutions. Mentally, at least, most of us could compile formidable lists of ``do's'' and ``don'ts''. The same old favourites recur year in year out with monotonous regularity. We resolve to get up earlier each morning, eat less, find more time to play with the children, do a thousand and one jobs about the house, be nice to people we don't like, drive carefully, and take the dog for a walk every day. Past experience has taught us that certain accomplishments are beyond attainment. If we remain inveterate smokers, it is only because we have so often experienced the frustration that results from failure. Most of us fail in our efforts at self-improvement because our schemes are too ambitious and we never have time to carry them out. We also make the fundamental error of announcing our resolutions to everybody so that we look even more foolish when we slip back into our bad old ways. Aware of these pitfalls, this year I attempted to keep my resolutions to myself. I limited myself to two modest ambitions: to do physical exercises every morning and to read more of an evening. An all-night party on New Year's Eve, provided me with a good excuse for not carrying out either of these new resolutions on the first day of the year, but on the second, I applied myself assiduously to the task. The daily exercises lasted only eleven minutes and I proposed to do them early in the morning before anyone had got up. The self-discipline required to drag myself out of bed eleven minutes earlier than usual was considerable. Nevertheless, I managed to creep down into the living-room for two days before anyone found me out. After jumping about on the carpet and twisting the human frame into uncomfortable positions, I sat down at the breakfast table in an exhausted condition. It was this that betrayed me. The next morning the whole family trooped in to watch the performance. That was really unsettling but I fended off the taunts and jibes of the family good-humouredly and soon everybody got used to the idea. However, my enthusiasm waned. The time I spent at exercises gradually diminished. Little by little the eleven minutes fell to zero. By January 10th, I was back to where I had started from. I argued that if I spent less time exhausting myself at exercises in the morning I would keep my mind fresh for reading when I got home from work. Resisting the hypnotizing effect of television, I sat in my room for a few evenings with my eyes glued to a book, one night, however, feeling cold and lonely, I went downstairs and sat in front of the television pretending to read. That proved to be my undoing, for I soon got back to my old bad habit of dozing off in front of the screen. I still haven't given up my resolution to do more reading. In fact, I have just bought a book entitled ``How to Read a Thousand Words a Minute''. Perhaps it will solve my problem, but I just haven't had time to read it! 
\subsection{Automation}
\label{sec-1-51}

One of the greatest advances in modern technology has been the invention of computers. They are already widely used in industry and in universities and the time may come when it will be possible for ordinary people to use them as well. Computers are capable of doing extremely complicated work in all branches of learning. They can solve the most complex mathematical problems or put thousands of unrelated facts in order. These machines can be put to varied uses. For instance, they can provide information on the best way to prevent traffic accidents, or they can count the number of times the word ``and'' has been used in the Bible. Because they work accurately and at high speeds, they save research workers years of hard work. This whole process by which machines can be used to work for us has been called automation. In the future, automation may enable human beings to enjoy far more leisure than they do today. The coming of automation is bound to have important social consequences. 

Some time ago an expert, on automation, Sir Leon Bagrit, pointed out that it was a mistake to believe that these machines could ``think''. There is no possibility that human beings will be ``controlled by machines''. Though computers are capable of learning from their mistakes and improving on their performance they need detailed instructions from human beings in order to be able to operate. They can never, as it were, lead independent lives, or ``rule the world'' by making decisions of their own.

Sir Leon said that in the future, computers would be developed which would be small enough to carry in the pocket. Ordinary people would then be able to use them to obtain valuable information. Computers could be plugged into a national network and be used like radios. For instance, people going on holiday could be informed about weather conditions; car drivers could be given alternative routes when there are traffic jams. It will also be possible to make tiny translating machines. This will enable people who do not share a common language to talk to each other without any difficulty or to read foreign publications. It is impossible to assess the importance of a machine of this sort, for many international misunderstandings are caused simply through our failure to understand each other. Computers will also be used in hospitals. By providing a machine with a patient's symptoms, a doctor will be able to diagnose the nature of his illness. Similarly, machines could be used to keep a check on a patient's health record and bring it up to date. Doctors will therefore have immediate access to a great many facts which will help them in their work. Book-keepers and accountants, too, could be relieved of dull clerical work, for the tedious task of compiling and checking lists of figures could be done entirely by machines. Computers are the most efficient servants man has ever had and there is no limit to the way they can be used to improve our lives. 
\subsection{Mud is mud}
\label{sec-1-52}

My cousin, Harry, keeps a large curiously shaped bottle on permanent display in his study. Despite the fact that the bottle is tinted a delicate shade of green, an observant visitor would soon notice that it is filled with what looks like a thick greyish substance. If you were to ask Harry what was in the bottle, he would tell you that it contained perfumed mud. If you expressed doubt or surprise, he would immediately invite you to smell it and then to rub some into your skin. This brief experiment would dispel any further doubts you might entertain. The bottle really does contain perfumed mud. How Harry came into the possession of this outlandish stuff makes an interesting story which he is fond of relating. Further more, the acquisition of this bottle cured him of a bad habit he had been developing for years. 

Harry used to consider it a great joke to go into expensive cosmetic shops and make outrageous requests for goods that do not exist. He would invent fanciful names on the spot. On entering a shop, he would ask for a new perfume called ``Scented Shadow'' or for ``insoluble bath cubes''. If a shop girl told him she had not heard of it, he would pretend to be considerably put out. He loved to be told that one of his imaginary products was temporarily out of stock and he would faithfully promise to call again at some future date, but of course he never did. How Harry managed to keep a straight face during these performances is quite beyond me. 

Harry does not need to be prompted to explain how he bought his precious bottle of mud. One day, he went to an exclusive shop in London and asked for ``Myrolite''. The shop assistant looked puzzled and Harry repeated the word, slowly stressing each syllable. When the girl shook her head in bewilderment, Harry went on to explain that ``myrolite'' was a hard, amber-like substance which could be used to remove freckles. This explanation evidently conveyed something to the girl who searched shelf after shelf. She produced all sorts of weird concoctions, but none of them met with Harry's requirements. When Harry put on his act of being mildly annoyed, the girl promised to order some for him. Intoxicated by his success, Harry then asked for perfumed mud. He expected the girl to look at him in blank astonishment. However, it was his turn to be surprised, for the girl's eyes immediately lit up and she fetched several botties which she placed on the counter for Harry to inspect. For once, Harry had to admit defeat. He picked up what seemed to be the smallest bottle and discreetly asked the price. He was glad to get away with a mere five guineas and he beat a hasty retreat, clutching the precious bottle under his arm. From then on, Harry decided that this little game he had invented might prove to be expensive. The curious bottle which now adorns the bookcase in his study was his first and last purchase of rare cosmetics.
\subsection{In the public interest}
\label{sec-1-53}

The Scandinavian countries are much admired all over the world for their enlightened social policies. Sweden has evolved an excellent system for protecting the individual citizen from high-handed or incompetent public officers. The system has worked so well, that it has been adopted in other countries like Denmark, Norway, Finland, and New Zealand. Even countries with large populations like Britain and the United States are seriously considering imitating the Swedes.

The Swedes were the first to recognize that public officials like civil servants, collectors can make mistakes or act over-zealously in the belief that they are serving the public. As long ago as 1809, the Swedish Parliament introduced a scheme to safeguard the interest of the individual. A parliamentary committee representing all political parties appoints a person who is suitably qualified to investigate private grievances against the State. The official title of the person is ``Justiteombudsman'', but the Swedes commonly refer to him as the ``J.O.'' or ``Ombudsman''. The Ombudsman is not subject to political pressure. He investigates complaints large and small that come to him from all levels of society. As complaints must be made in writing, the Ombudsman receives an average of 1200 letters a year. He has eight lawyer assistants to help him and he examines every single letter in detail. There is nothing secretive about the Ombudsman's work, for his correspondence is open to public inspection. If a citizen's complaint is justified, the Ombudsman will act on his behalf. The action he takes varies according to the nature of the complaint. He may gently reprimand an official or even suggest to parliament that a law be altered. The following case is a typical example of the Ombudsman's work. 

A foreigner living in a Swedish village wrote to the Ombudsman complaining that he had been ill-treated by the police, simply because he was a foreigner. The Ombudsman immediately wrote to the Chief of Police in the district asking him to send a record of the case. There was nothing in the record to show that the foreigner's complaint was justified and the Chief of Police stoutly denied the accusation. It was impossible for the Ombudsman to take action, but when he received a similar complaint from another foreigner in the same village, he immediately sent one of his lawyers to investigate the matter. The lawyer ascertained that a policeman had indeed dealt roughly with foreigners on several occasions. The fact that the policeman was prejudiced against foreigners could not be recorded in he official files. It was only possible for the Ombudsman to find this out by sending one of his representatives to check the facts. The policeman in question was severely reprimanded and was informed that if any further complaints were lodged against him, he would be prosecuted. The Ombudsman's prompt action at once put an end to an unpleasant practice which might have gone unnoticed.
\subsection{Instinct or cleverness?}
\label{sec-1-54}

We have been brought up to fear insects. We regard them as unnecessary creatures that do more harm than good. Man continually wages war on item, for they contaminate his food, carry diseases, or devour his crops. They sting or bite without provocation; they fly uninvited into our rooms on summer nights, or beat against our lighted windows. We live in dread not only of unpleasant insects like spiders or wasps, but of quite harmless ones like moths. Reading about them increases our understanding with out dispelling our fears. Knowing that the industrious ant lives in a highly organized society does nothing to prevent us from being filled with revulsion when we find hordes of them crawling over a carefully prepared picnic lunch. No matter how much we like honey, or how much we have read about the uncanny sense of direction which bees possess, we have a horror of being stung. Most of our fears are unreasonable, but they are impossible to erase. At the same time, however, insects are strangely fascinaing. We enjoy reading about them, especially when we find that, like the praying mantis, they lead perfectly horrible lives. We enjoy staring at them entranced as they go about their business, unaware (we hope) of our presence. Who has not stood in awe at the sight of a spider pouncing on a fly, or a column of ants triumphantly bearing home an enormous dead beetle ? 

Last summer I spent days in the garden watching thousands of ants crawling up the trunk of my prize peach tree. The tree has grown against a warm wall on a sheltered side of the house. I am especially proud of it, not only because it has survived several severe winters, but because it occasionally produces luscious peaches. During the summer, I noticed that the leaves of the tree were beginning to wither. Clusters of tiny insects called aphides were to be found on the underside of the leaves. They were visited by a laop colony of ants which obtained a sort of honey from them. I immediately embarked on an experiment which, even though it failed to get rid of the ants, kept me fascinated for twenty-four hours. I bound the base of the tree with sticky tape , making it impossible for the ants to reach the aphides. The tape was so sticky that they did not dare to cross it. For a long time, I watched them scurrying around the base of the tree in bewilderment. I even went out at midnight with a torch and noted with satisfaction (and surprise) that the ants were still swarming around the sticky tape without being able to do anything about it. I got up early next morning hoping to find that the ants had given up in despair. Instead, I saw that they had discovered a new route. They were climbing up the wall of the house and then on to the leaves of the tree. I realized sadly that I had been completely defeated by their ingenuity. The ants had been quick to find an answer to my thoroughly unscientific methods!
\subsection{From the earth: Greetings}
\label{sec-1-55}

Radio astronomy has greatly increased our understanding of the universe. Radio telescopes have one big advantage over conventional telescopes in that they can operate in all weather conditions and can pick up signals coming from very distant stars. These signals are produced by colliding stars or nuclear reactions in outer space. The most powerful signals that have been received have been emitted by what seem to be truly colossal stars which scientists have named ``quasars''.

A better understanding of these phenomena may completely alter our conception of the nature of the universe. The radio telescope at Jodrell Bank in England was for many years the largest in the world. A new telescope, over twice the size, was recently built at Sugar Grove in West Virginia. Astronomers no longer regard as fanciful the idea that they may one day pick up signals which have been sent by intelligent beings on other worlds. This possibility gives rise to interesting speculations. Highly advanced civilizations may have existed on other planets long before intelligent forms of life evolved on the earth. Conversely, intelligent being which are just beginning to develop on remote worlds may be ready to pick up our signals in thousands of years' time, or when life on earth has become extinct. Such speculations no longer belong to the realm of science fiction, for astronomers are now exploring the chances of communicating with living creatures (if they exist) on distant planets. This undertaking which has been named Project Ozma was begun in 1960, but it may take a great many years before results are obtained. 

Aware of the fact that it would be impossible to wait thousands or millions of years to receive an answer from a distant planet, scientists engaged in Project Ozma are concentrating their attention on stars which are relatively close. One of the most likely stars is Tau Ceti which is eleven light years away. If signals from the earth were received by intelligent creatures on a planet circling this star, we would have to wait twenty-two years for an answer. The Green Bank telescope in West Virginia has been specially designed to distinguish between random signals and signals which might be in code. Even if contact were eventually established, astronomers would not be able to rely on language to communicate with other beings. They would use mathematics as this is the only truly universal language. Numbers have the same value anywhere. For this reason, intelligent creatures in any part of the universe would be able to understand a simple arithmetical sequence. They would be able to reply to our signals using similar methods. The next step would be to try to develop means for sending television pictures. A single picture would tell us more than thousands of words. In an age when anything seems to be possible, it would be narrow-minded in the extreme to ridicule these attempts to find out if there is life in other parts of the universe. 
\subsection{The river beside our farm}
\label{sec-1-56}

The river which forms the eastern boundary of our farm has always played an important part in our lives. Without it we could not make a living. There is only enough spring water to supply the needs of the house, so we have to pump from the river for farm use. We tell the river all our secrets. We know instinctively, just as beekeepers with their bees, that misfortune might overtake us if the important events of our lives were not related to it.

We have special river birthday parties in the summer. Sometimes we go up-stream to a favourite backwater, sometimes we have our party at the boathouse, which a predecessor of ours at the farm built in the meadow hard by the deepest pool for swimming and diving. In a heat-wave we choose a midnight birthday party and that is the most exciting of all. We welcome the seasons by the river side, crowning the youngest girl with flowers in the spring, holding a summer festival on Midsummer Eve, giving thanks for the harvest in the autumn, and throwing a holly wreath into the current in the winter. After a long period of rain the river may overflow its banks. This is a rare occurrence as our climate seldom god to extremes. We are lucky in that only the lower fields, which make up a very small proportion of our farm, are affected by flooding, but other farms are less favourably sited, and flooding can sometimes spell disaster for their owners. 

One bad winter we watched the river creep up the lower meadows. All the cattle had been moved into stalls and we stood to lose little. We were, however, worried about our nearest neighbours, whose farm was low lying and who were newcomers to the district. As the floods had put the telephone out of order, we could not find out how they were managing. From an attic window we could get a sweeping view of the river where their land joined ours, and at the most critical juncture we took turns in watching that point. The first sign of disaster was a dead sheep floating down. Next came a horse, swimming bravely, but we were afraid that the strength of the current would prevent its landing anywhere before it became exhausted. Suddenly a raft appeared, looking rather like Noah's ark, carrying the whole family, a few hens, the dogs, a cat, and a bird in a cage. We realized that they must have become unduly frightened by the rising flood, for their house, which had sound foundations, would have stood stoutly even if it had been almost submerged. The men of our family waded down through our flooded meadows with boathooks, in the hope of being able to grapple a corner of the raft and pull it out of the current towards our bank. We still think it a miracle that they were able to do so.
\subsection{The return of the native}
\label{sec-1-57}

I stopped to let the car cool off and to study the map. I had expected to be near my objective by now, but everything still seemed alien to me. I was only five when my father had taken me abroad, and that was eighteen years ago. When my mother had died after a tragic accident, he did not quickly recover from the shock and loneliness. Everything around him was full of her presence, continually re-opening the wound. So he decided to emigrate. In the new country he became absorbed in making a new life for the two of us, so that he gradually ceased to grieve. He did not marry again and I was brought up without a woman's care; but I lacked for nothing, for he was both father and mother to me. He always meant to go back one day but not to stay. His roots and mine had become too firmly embedded in the new land. But he wanted to see the old folk again and to visit my mother's grave. He became mortally ill a few months before we had planned to go and, when he knew that he was dying, he made me promise to go on my own.

I hired a car the day after landing and bought a comprehensive book of maps, which I found most helpful on the cross country journey, but which I did not think I should need on the last stage. It was not that I actually remembered anything at all. But my father had described over and over again what we should see at every milestone, after leaving the nearest town, so that I was positive I should recognize it as familiar territory. Well, I had been wrong, for I was now lost.

I looked at the map and then at the milometer. I had come ten miles since leaving the town, and at this point, according to my father, I should be looking at farms and cottages in a valley, with the spire of the church of our village showing in the far distance. I could see no valley, no farms, no cottages and no church spire---only a lake. I decided that I must have taken a wrong turning somewhere. So I drove back to the town and began to retrace the route, taking frequent glances at the map. I landed up at the same corner. The curious thing was that the lake was not marked on the map. I felt as if I had stumbled into a nightmare country, as you sometimes do in dreams. And, as in a nightmare, there was nobody in sight to help me. Fortunately for me, as I was wondering what to do next, there appeared on the horizon a man on horseback, riding in my direction. I waited till he came near, then I asked him the way to our old village. He said that there was now no village. I thought he must have misunderstood me. so I repeated its name. This time he pointed to the lake. The village no longer existed because it had been submerged, and all the valley too. The lake was not a natural one, but a man made reservoir. 
\subsection{A little spot of bother}
\label{sec-1-58}

The old lady was glad to be back at the block of flats where she lived. Her shopping had tired her and her basket had grown heavier with every step of the way home. In the lift her thoughts were on lunch and a good rest; but when she got out at her own floor, both were forgotten in her sudden discovery that her front door was open. She was thinking that she must reprimand her daily maid the next morning for such a monstrous piece of negligence, when she remembered that she had gone shopping after the maid had left and she knew that she had turned both keys in their locks. She walked slowly into the hall and at once noticed that all the room doors were open, yet following her regular practice she had shut them before going out. Looking into the drawing room, she saw a scene of confusion over by her writing desk. It was as clear as daylight then that burglars had forced an entry during her absence. Her first impulse was to go round all the rooms looking for the thieves, but then she decided that at her age it might be more prudent to have someone with her, so she went to fetch the porter from his basement. By this time her legs were beginning to tremble, so she sat down and accepted a cup of very strong tea, while he telephoned the police. Then, her composure regained, she was ready to set off with the porter's assistance to search for any intruders who might still be lurking in her flat. 

They went through the rooms, being careful to touch nothing, as they did not want to hinder the police in their search for fingerprints. The chaos was inconceivable. She had lived in the flat for thirty years and was a veritable magpie at hoarding; and it seemed as though everything she possessed had been tossed out and turned over and over. At least sorting out the things she should have discarded years ago was now being made easier for her. Then a police inspector arrived with a constable and she told them of her discovery of the ransacked flat. The inspector began to look for fingerprints, while the constable checked that the front door locks had not been forced, thereby proving that the burglars had either used skeleton keys or entered over the balcony. There was no trace of fingerprints, but the inspector found a dirty red bundle that contained jewellery which the old lady said was not hers. So their entry into this flat was apparently not the burglars' first job that day and they must have been disturbed. The inspector then asked the old lady to try to check what was missing by the next day and advised her not to stay alone in the flat for a few nights. The old lady thought he was a fussy creature, but since the porter agreed with him, she rang up her daughter and asked for her help in what she described as a little spot of bother. 
\subsection{Possession amassing and collecting}
\label{sec-1-59}

People tend to amass possessions, sometimes without being aware of doing so. Indeed they can have a delightful surprise when they find something useful which they did not know they owned. Those who never have to change house become indiscriminate collectors of what can only be described as clutter. They leave unwanted objects in drawers, cupboards and attics for years, in the belief that they may one day need just those very things. As they grow old, people also accumulate belongings for two other reasons, lack of physical and mental energy, both of which are essential in turning out and throwing away, and sentiment. Things owned for a long time are full of associations with the past, perhaps with relatives who are dead, and so they gradually acquire a value beyond their true worth. Some things are collected deliberately in the home in an attempt to avoid waste. Among these I would list string and brown paper, kept by thrifty people when a parcel has been opened, to save buying these two requisites. Collecting small items can easily become a mania. I know someone who always cuts out from newspapers sketches of model clothes that she would like to buy, if she had the money. As she is not rich, the chances that she will ever be able to afford such purchases are remote; but she is never sufficiently strongrminded to be able to stop the practice. It is a harmless habit, but it litters up her desk to such an extent that every time she opens it, loose bits of paper fall out in every direction.

Collecting as a serious hobby is quite different and has many advantages. It provides relaxation for leisure hours, as just looking at one's treasures is always a joy. One does not have to go outside for amusement, since the collection is housed at home. Whatever it consists of , stamps, records, first editions of books, china, glass, antique furniture, pictures, model cars, stuffed birds, toy animals, there is always something to do in connection with it, from finding the right place for the latest addition to verifying facts in reference books. This hobby educates one not only in the chosen subject, but also in general matters which have some bearing on it. There are also other benefits. One wants to meet like-minded collectors, to get advice, to compare notes, to exchange articles, to show off the latest find. So one's circle of friends grows. Soon the hobby leads to travel, perhaps to a meeting in another town, possibly a trip abroad in search of a rare specimen, for collectors are not confined to any one country. Over the years one may well become an authority on one's hobby and will very probably be asked to give informal talks to little gatherings and then, if successful, to larger audiences. In this way self-confidence grows, first from mastering a subject, then from being able to talk about it. Collecting, by occupying spare time so constructively, makes a person contented, with no time for boredom. 
\subsection{The importance of Punctuality}
\label{sec-1-60}

Punctuality is a necessary habit in all public affairs of a civilized society. Without it, nothing could ever be brought to a conclusion; everything would be in a state of chaos. Only in a sparsely populated rural community is it possible to disregard it. In ordinary living there can be some tolerance of unpunctuality. The intellectual, who is working on some abstruse problem, has everything coordinated and organized for the matter in hand. He is therefore forgiven, if late for a dinner party. But people are often reproached for unpunctuality when their only fault is cutting things fine. It is hard for energetic, quick-minded people to waste time, so they are often tempted to finish a job before setting out to keep an appointment. If no accidents occur on the way, like punctured tyres, diversions of traffic, sudden descent of fog, they will be on time. They are often more industrious, useful citizens than those who are never late. The over-punctual can be as much a trial to others as the unpunctual. The guest who arrives half an hour too soon is the greatest nuisance. Some friends of my family had this irritating habit. The only thing to do was ask them to come half an hour later than the other guests. Then they arrived just when we wanted them. 

If you are catching a train, it is always better to be comfortably early than even a fraction of a minute too late. Although being early may mean wasting a little time, this will be less than if you miss the train and have to wait an hour or more for the next one; and you avoid the frustration of arriving at the very moment when the train is drawing out of the station and being unable to get on it. An even harder situation is to be on the platform in good time for a train and still to see it go off without you. Such an experience befell a certain young girl the first time she was travelling alone.

She entered the station twenty minutes before the train was due, since her parents had impressed upon her that it would be unforgivable to miss it and cause the friends with whom she was going to stay to make two journeys to meet her. She gave her luggage to a porter and showed him her ticket. To her horror he said that she was two hours too soon. She felt in her handbag for the piece of paper on which her father had written down all the details of the journey and give it to the porter. He agreed that a train did come into the station at the time on the paper and that it did stop, but only to take on water, not passengers. The girl asked to see a timetable, feeling sure that her father could not have made such a mistake. The porter went to fetch one and arrived back with the stationmaster, who produced it with a flourish and pointed out a microscopic ``o'' beside the time of the arrival of the train at his station; this little ``o'' indicated that the train only stopped for water. Just as that moment the train came into the station. The girl, tears streaming down her face, begged to be allowed to slip into the guard's van. But the stationmaster was adamant: rules could not be broken. And she had to watch that train disappear towards her destination while she was left behind.
\section{NCE 4}
\label{sec-2}
\subsection{Finding fossil man}
\label{sec-2-1}

We can read of things that happened 5,000 years ago in the Near East, where people first learned to write. But there are some parts of the word where even now people cannot write. The only way that they can preserve their history is to recount it as sagas---legends handed down from one generation of another. These legends are useful because they can tell us something about migrations of people who lived long ago, but none could write down what they did. Anthropologists wondered where the remote ancestors of the Polynesian peoples now living in the Pacific Islands came from. The sagas of these people explain that some of them came from Indonesia about 2,000 years ago.

But the first people who were like ourselves lived so long ago that even their sagas, if they had any, are forgotten. So archaeologists have neither history nor legends to help them to find out where the first ``modern men'' came from.

Fortunately, however, ancient men made tools of stone, especially flint, because this is easier to shape than other kinds. They may also have used wood and skins, but these have rotted away. Stone does not decay, and so the tools of long ago have remained when even the bones of the men who made them have disappeared without trace.
\subsection{Spare that spider}
\label{sec-2-2}

Why, you may wonder, should spiders be our friends? Because they destroy so many insects, and insects include some of the greatest enemies of the human race. Insects would make it impossible for us to live in the world; they would devour all our crops and kill our flocks and herds, if it were not for the protection we get from insect-eating animals. We owe a lot to the birds and beasts who eat insects but all of them put together kill only a fraction of the number destroyed by spiders. Moreover, unlike some of the other insect eaters, spiders never do the harm to us or our belongings.

Spiders are not insects, as many people think, nor even nearly related to them. One can tell the difference almost at a glance, for a spider always has eight legs and insect never more than six.

How many spiders are engaged in this work no our behalf? One authority on spiders made a census of the spiders in grass field in the south of England, and he estimated that there were more than 2,250,000 in one acre; that is something like 6,000,000 spiders of different kinds on a football pitch. Spiders are busy for at least half the year in killing insects. It is impossible to make more than the wildest guess at how many they kill, but they are hungry creatures, not content with only three meals a day. It has been estimated that the weight of all the insects destroyed by spiders in Britain in one year would be greater than the total weight of all the human beings in the country.
\subsection{Matterhorn man}
\label{sec-2-3}

Modern alpinists try to climb mountains by a route which will give them good sport, and the more difficult it is, the more highly it is regarded. In the pioneering days, however, this was not the case at all. The early climbers were looking for the easiest way to the top, because the summit was the prize they sought, especially if it and never been attained before. It is true that during their explorations they often faced difficulties and dangers of the most perilous nature, equipped in a manner with would make a modern climber shudder at the thought, but they did not go out of their way to court such excitement. They had a single aim, a solitary goal---the top!

It is hard for us to realize nowadays how difficult it was for the pioneers. Except for one or two places such as Zermatt and Chamonix, which had rapidly become popular, Alpine village tended to be impoverished settlements cut off from civilization by the high mountains. Such inns as there were generally dirty and flea-ridden; the food simply local cheese accompanied by bread often twelve months old, all washed down with coarse wine. Often a valley boasted no inn at all, and climbers found shelter wherever they could---sometimes with the local priest (who was usually as poor as his parishioners), sometimes with shepherds or cheese-makers. Invariably the background was the same: dirt and poverty, and very uncomfortable. For men accustomed to eating seven-course dinners and sleeping between fine linen sheets at home, the change to the Alps must have very hard indeed.
\subsection{Seeing hands}
\label{sec-2-4}

Several cases have been reported in Russia recently of people who can detect colours with their fingers, and even see through solid and walls. One case concerns and eleven-year-old schoolgirl, Vera Petrova, who has normal vision but who can also perceive things with different parts of her skin, and through solid walls. This ability was first noticed by her father. One day she came into his office and happened to put her hands on the door of a locked safe. Suddenly she asked her father why he kept so many old newspapers locked away there, and even described the way they were done up in bundles.

Vera's curious talent was brought to the notice of a scientific research institute in the town of Ulyanovsk, near where she lives, and in April she was given a series of tests by a special commission of the Ministry of Health of the Russian Federal Republic. During these tests she was able to read a newspaper through an opaque screen and, stranger still, by moving her elbow over a child's game of Lotto she was able to describe the figures and colours printed on it; and, in another instance, wearing stockings and slippers, to make out with her foot the outlines and colours of a picture hidden under a carpet. Other experiments showed that her knees and shoulders had a similar sensitivity. During all these tests Vera was blindfold; and, indeed, except when blindfold she lacked the ability to perceive things with her skin. It was also found that although she could perceive things with her fingers this ability ceased the moment her hands were wet.
\subsection{Youth}
\label{sec-2-5}

People are always talking about ``the problem of youth''. If there is one---which I take leave to doubt---then it is older people who create it, not the young themselves. Let us get down to fundamentals and agree that the young are after all human beings---people just like their elders. There is only one difference between an old man and a young one: the young man has a glorious future before him and the old one has a splendid future behind him: and maybe that is where the rub is.

When I was a teenager, I felt that I was just young and uncertain---that I was a new boy in a huge school, and I would have been very pleased to be regarded as something so interesting as a problem. For one thing, being a problem gives you a certain identity, and that is one of the things the young are busily engaged in seeking.

I find young people exciting. They have an air of freedom, and they not a dreary commitment to mean ambitions or love of comfort. They are not anxious social climbers, and they have no devotion to material things. All this seems to me to link them with life, and the origins of things. It's as if they were, in some sense, cosmic beings in violent and lovely contrast with us suburban creatures. All that is in my mind when I meet a young person. He may be conceited, ill-mannered, presumptuous or fatuous, but I do not turn for protection to dreary cliches about respect of elders---as if mere age were a reason for respect. I accept that we are equals, and I will argue with him, as an equal, if I think he is wrong.
\subsection{The sporting spirit}
\label{sec-2-6}

I am always amazed when I hear people saying that sport creates goodwill between the nations, and that if only the common peoples of the would could meet one another at football or cricket, they would have no inclination to meet on the hattlefield. Even if one didn't know from concrete examples (the 1936 Olympic Games, for instance) that international sporting contests lead to orgies of hatred, one could deduce if from general principles.

Nearly all the sports practised nowadays are competitive. You play to win, and the game has little meaning unless you do your utmost to win. On the village green, where you pick up sides and no feeling of local patriotism is involved, it is possible to play simply for the fun and exercise: but as soon as a the question of prestige arises, as soon as you feel that you and some larger unit will be disgraced if you lose, the most savage combative instincts are aroused. Anyone who has played even in a school football match knows this. At the international level, sport is frankly mimic warfare. But the significant thing is not the behaviour of the players but the attitude of the spectators: and, behind the spectators, of the nations who work themselves into furies over these absurd contests, and seriously believe---at any rate for short periods---that running, jumping and kicking a ball are tests of national virtue.
\subsection{Bats}
\label{sec-2-7}

Not all sounds made by animals serve as language, and we have only to turn to that extraordinary discovery of echo-location in bats to see a case in which the voice plays a strictly utilitarian role.

To get a full appreciation of what this means we must turn first to some recent human inventions. Everyone knows that if he shouts in the vicinity of a wall or a mountainside, an echo will come back. The further off this solid obstruction, the longer time will elapse for the return of the echo. A sound made by tapping on the hull of a ship will be reflected from the sea bottom, and by measuring the time interval between the taps and the receipt of the echoes, the depth of the sea at that point can be calculated. So was born the echo-sounding apparatus, now in general use in ships. Every solid object will reflect a sound, varying according to the size and nature of the object. A shoal of fish will do this. So it is a comparatively simple step from locating the sea bottom to locating a shoal of fish. With experience, and with improved apparatus, it is now possible not only to locate a shoal but to tell if it is herring, cod, or other well-known fish, by the pattern of its echo.

It has been found that certain bats emit squeaks and by receiving the echoes, they can locate and steer clear of obstacles---or locate flying insects on which they feed. This echo-location in bats is often compared with radar, the principle of which is similar.
\subsection{Trading standards}
\label{sec-2-8}

Chickens slaughtered in the United States, claim officials in Brussels, are not fit to grace European tables. No, say the American: our fowl are fine, we simply clean them in a different way. These days, it is differences in national regulations, far more than tariffs, that put sand in the wheels of trade between rich countries. It is not just farmers who are complaining. An electric razor that meets the European Union's safety standards must be approved by American testers before it can be sold in the United States, and an American-made dialysis machine needs the EU's okay before is hits the market in Europe.

As it happens, a razor that is safe in Europe is unlikely to electrocute Americans. So, ask businesses on both sides of the Atlantic, why have two lots of tests where one would do? Politicians agree, in principle, so America and the EU have been trying to reach a deal which would eliminate the need to double-test many products. They hope to finish in time for a trade summit between America and the EU on May 28TH. Although negotiators are optimistic, the details are complex enough that they may be hard-pressed to get a deal at all.

Why? One difficulty is to construct the agreements. The Americans would happily reach one accord on standards for medical devices and them hammer out different pacts covering, say, electronic goods and drug manufacturing. The EU---following fine continental traditions---wants agreement on general principles, which could be applied to many types of products and perhaps extended to other countries.
\subsection{Royal espionage}
\label{sec-2-9}

Alfred the Great acted his own spy, visiting Danish camps disguised as a minstrel. In those days wandering minstrels were welcome everywhere. They were not fighting men, and their harp was their passport. Alfred had learned many of their ballads in his youth, and could vary his programme with acrobatic tricks and simple conjuring.

While Alfred's little army slowly began to gather at Athelney, the king himself set out to penetrate the camp of Guthrum, the commander of the Danish invaders. There had settled down for the winter at Chippenham: thither Alfred went. He noticed at once that discipline was slack: the Danes had the self-confidence of conquerors, and their security precautions were casual. They lived well, on the proceeds of raids on neighbouring regions. There they collected women as well as food and drink, and a life of ease had made them soft.

Alfred stayed in the camp a week before he returned to Athelney. The force there assembled was trivial compared with the Danish horde. But Alfred had deduced that the Danes were no longer fit for prolonged battle: and that their commissariat had no organization, but depended on irregular raids.

So, faced with the Danish advance, Alfred did not risk open battle but harried the enemy. He was constantly on the move, drawing the Danes after him. His patrols halted the raiding parties: hunger assailed the Danish army. Now Alfred began a long series of skirmishes---and within a month the Danes had surrendered. The episode could reasonably serve as a unique epic of royal espionage!
\subsection{Silicon valley}
\label{sec-2-10}

Technology trends may push Silicon Valley back to the future. Carver Mead, a pioneer in integrated circuits and a professor of computer science at the California Institute of Technology, notes there are now work-stations that enable engineers to design, test and produce chips right on their desks, much the way an editor creates a newsletter on a Macintosh. As the time and cost of making a chip drop to a few days and a few hundred dollars, engineers may soon be free to let their imaginations soar without being penalized by expensive failures. Mead predicts that inventors will be able to perfect powerful customized chips over a weekend at the office---spawning a new generation of garage start-ups and giving the U.S. a jump on its foreign rivals in getting new products to market fast. ``We're got more garages with smart people,'' Mead observes. ``We really thrive on anarchy.''

And on Asians. Already, orientals and Asian Americans constitute the majority of the engineering staffs at many Valley firms. And Chinese, Korean, Filipino and Indian engineers are graduating in droves from California's colleges. As the heads of next-generation start-ups, these Asian innovators can draw on customs and languages to forge righter links with crucial Pacific Rim markets. For instance, Alex Au, a Stanford Ph. D. from Hong Kong, has set up a Taiwan factory to challenge Japan's near lock on the memory-chip market. India-born N.Damodar Reddy's tiny California company reopened an AT \& T chip plant in Kansas City last spring with financing from the state of Missouri. Before it becomes a retirement village, Silicon Valley may prove a classroom for building a global business.
\subsection{How to grow old}
\label{sec-2-11}

Some old people are oppressed by the fear of death. In the young there is a justification for this feeling. Young men who have reason to fear that they will be killed in battle may justifiably feel bitter in the thought that they have cheated of the best things that life has to offer. But in an old man who has known human joys and sorrows, and has achieved whatever work it was in him to do, the fear of death is somewhat abject and ignoble. The best way to overcome it---so at least it seems to me---is to make your interests gradually wider and more impersonal, until bit by bit the walls of the ego recede, and your life becomes increasingly merged in the universal life. An individual human existence should be like a river---small at first, narrowly contained within its banks, and rushing passionately past boulders and over waterfalls. Gradually the river grows wider, the banks recede, the waters flow more quietly, and in the end, without any visible break, they become merged in the sea, and painlessly lose their individual being. The man who, in old age, can see his life in this way, will not suffer from the fear of death, since the things he cares for will continue. And if, with the decay of vitality, weariness increases, the thought of rest will be not unwelcome. I should wish to die while still at work, knowing that others will carry on what I can no longer do, and content in the thought that what was possible has been done.
\subsection{Banks and their customers}
\label{sec-2-12}

When anyone opens a current account at a bank, he is lending the bank money, repayment of which he may demand at any time, either in cash or by drawing a cheque in favour of another person. Primarily, the banker-customer relationship is that of debtor and creditor---who is which depending on whether the customer's account is in credit or is overdrawn. But, in addition to that basically simple concept, the bank and its customer owe a large number of obligations to one another. Many of these obligations can give in to problems and complications but a bank customer, unlike, say, a buyer of goods, cannot complain that the law is loaded against him.

The bank must obey its customer's instructions, and not those of anyone else. When, for example, a customer first opens an account, he instructs the bank to debit his account only in respect of cheques draw by himself. He gives the bank specimens of his signature, and there is a very firm rule that the bank has no right or authority to pay out a customer's money on a cheques on which its customer's signature has been forged. It makes no difference that the forgery may have been a very skilful one: the bank must recognize its customer's signature. For this reason there is no risk to the customer in the practice, adopted by banks, of printing the customer's name on his cheques. If this facilitates forgery, it is the bank which will lose, not the customer.
\subsection{The search for oil}
\label{sec-2-13}

The deepest holes of all made for oil, and they go down to as much as 25,0000 feet. But we not need to send men down to get the oil our, as we must with other mineral deposits. The holes are only borings, less than a foot in diameter. My particular experience is largely in oil, and the search for oil has done more to improve deep drilling than any other mining activity. When is has been decided where we are going to drill, we put up at the surface an oil derrick. It has to be tall because it is like a giant block and tackle, and we have to lower into the ground and haul out of the ground great lengths of drill pipe which are rotated by an engine at the top and are fitted with a cutting bit at the bottom.

The geologist needs to know what rocks the drill has reached, so every so often a sample is obtained with a coring bit. It cuts a clean cylinder of rock, from which can be seen the strata the drill has been cutting through. Once we get down to the oil, it usually flows to the surface because great pressure, either from or water, is pushing it. This pressure must be under control, and we control it by means of the mud which we circulate down the drill pipe. We endeavour to avoid the old, romantic idea of a gusher, which wastes oil and gas. We want it to stay down the hole until we can lead it off in a controlled manner.
\subsection{The Butterfly Effect}
\label{sec-2-14}

Beyond two or three days, the world's best weather forecasts are speculative, and beyond six or seven they are worthless.

The Butterfly Effect is the reason. For small pieces of weather---and to a global forecaster, small can mean thunderstorms and blizzards---any prediction deteriorates rapidly. Errors and uncertainties multiply, cascading upward through a chain of turbulent features, from dust devils and squalls up to continent-size eddies that only satellites can see.

The modern weather models work with a grid of points of the order of sixty miles apart, and even so, some starting data has to guessed, since ground stations and satellites cannot see everywhere. But suppose the earth could be covered with sensors spaced one foot apart, rising at one-foot intervals all the way to the top of the atmosphere. Suppose every sensor gives perfectly accurate readings of temperature, pressure, humidity, and any other quantity a meteorologist would want. Precisely at noon an infinitely powerful computer takes all the data and calculates what will happen at each point at 12.01, then 1202, then 12.03\ldots{}

The computer will still be unable to predict whether Princeton, New Jersey, will have sun or rain on a day one month away. At noon the spaces between the sensors will hide fluctuations that the computer will not know about, tiny deviations from the average. By 12.01, those fluctuations will already have created small errors one foot away. Soon the errors will have multiplied to the ten-foot scale, and so on up to the size of the globe.
\subsection{Secrecy in industry}
\label{sec-2-15}

Two factors weigh heavily against the effectiveness of scientific research in industry. One is the general atmosphere of secrecy in which it is carried out, the other the lack of freedom of the individual research worker. In so far as any inquiry is a secret one, it naturally limits all those engaged in carrying it out from effective contact with their fellow scientists either in other countries or in universities, or even, often enough, in other departments of the same firm. The degree of secrecy naturally varies considerably. Some of the bigger firms are engaged in researches which are of such general and fundamental nature that it is a positive advantage to them not to keep them secret. Yet a great many processes depending on such research are sought for with complete secrecy until the stage at which patents can be taken out. Even more processes are never patented at all but kept as secret processes. This applies particularly to chemical industries, where chance discoveries play a much larger part than they do in physical and mechanical industries. Sometimes the secrecy goes to such an extent that the whole nature of the research cannot be mentioned. Many firms, for instance, have great difficulty in obtaining technical or scientific books from libraries because they are unwilling to have names entered as having taken out such and such a book, for fear the agents of other firms should be able to trace the kind of research they are likely to be undertaking.
\subsection{The modern city}
\label{sec-2-16}

In the organization of industrial life the influence of the factory upon the physiological and mental state of the workers has been completely neglected. Modern industry is based on the conception of the maximum production at lowest cost, in order that an individual or a group of individuals may earn as much money as possible. It has expanded without any idea of the true nature of the human beings who run the machines, and without giving any consideration to the effects produced on the individuals and on their descendants by the artificial mode of existence imposed by the factory. The great cities have been built with no regard for us. The shape and dimensions of the skyscrapers depend entirely on the necessity of obtaining the maximum income per square foot of ground, and of offering to the tenants offices and apartments that please them. This caused the construction of gigantic buildings where too large masses of human beings are crowded together. Civilized men like such a way of living. While they enjoy the comfort and banal luxury of their dwelling, they do not realize that they are deprived of the necessities of life. The modern city consists of monstrous edifices and of dark, narrow streets full of petrol fumes and toxic gases, torn by the noise of the taxicabs, lorries and buses, and thronged ceaselessly by great crowds. Obviously, it has not been planned for the good of its inhabitants.
\subsection{A man-made disease}
\label{sec-2-17}

In the early days of the settlement of Australia, enterprising settlers unwisely introduced the European rabbit. This rabbit had no natural enemies in the Antipodes, so that it multiplied with that promiscuous abandon characteristic of rabbits. It overran a whole continent. It caused devastation by burrowing and by devouring the herbage which might have maintained millions of sheep and cattle. Scientists discovered that this particular variety of rabbit (and apparently no other animal) was susceptible to a fatal virus disease, myxomatosis. By infecting animals and letting them loose in the burrows, local epidemics of this disease could be created. Later it was found that there was a type of mosquito which acted as the carrier of this disease and passed it on to the rabbits. So while the rest of the world was trying to get rid of mosquitoes, Australia was encouraging this one. It effectively spread the disease all over the continent and drastically reduced the rabbit population. It later became apparent that rabbits were developing a degree of resistance to this disease, so that the rabbit population was unlikely to be completely exterminated. There were hopes, however, that the problem of the rabbit would become manageable.

Ironically, Europe, which had bequeathed the rabbit as a pest to Australia, acquired this man-made disease as a pestilence. A French physician decided to get rid of the wild rabbits on his own estate and introduced myxomatosis. It did not, however, remain within the confines of his estate. It spread through France, Where wild rabbits are not generally regarded as a pest but as sport and a useful food supply, and it spread to Britain where wild rabbits are regarded as a pest but where domesticated rabbits, equally susceptible to the disease, are the basis of a profitable fur industry. The question became one of whether Man could control the disease he had invented.
\subsection{Porpoises}
\label{sec-2-18}

There has long been a superstition among mariners that porpoises will save drowning men by pushing them to the surface, or protect them from sharks by surrounding them in defensive formation. Marine Studio biologists have pointed out that, however intelligent they may be, it is probably a mistake to credit dolphins with any motive of lifesaving. On the occasions when they have pushed to shore an unconscious human being they have much more likely done it out of curiosity or for sport, as in riding the bow waves of a ship. In 1928 some porpoises were photographer working like beavers to push ashore a waterlogged mattress. If, as has been reported, they have protected humans from sharks, it may have been because curiosity attracted them and because the scent of a possible meal attracted the sharks. Porpoises and sharks are natural enemies. It is possible that upon such an occasion a battle ensued, with the sharks being driven away or killed.

Whether it be bird, fish or beast, the porpoise is intrigued with anything that is alive. They are constantly after the turtles, who peacefully submit to all sorts of indignities. One young calf especially enjoyed raising a turtle to the surface with his snout and then shoving him across the tank like an aquaplane. Almost any day a young porpoise may be seen trying to turn a 300-pound sea turtle over by sticking his snout under the edge of his shell and pushing up for dear life. This is not easy, and may require two porpoises working together. In another game, as the turtle swims across the oceanarium, the first porpoise swoops down from above and butts his shell with his belly. This knocks the turtle down several feet. He no sooner recovers his equilibrium than the next porpoise comes along and hits him another crack. Eventually the turtle has been butted all the way down to the floor of the tank. He is now satisfied merely to try to stand up, but as soon as he does so a porpoise knocks him flat. The turtle at last gives up by pulling his feet under his shell and the game is over.
\subsection{The stuff of dreams}
\label{sec-2-19}

It is fairly clear that sleeping period must have some function, and because there is so much of it the function would seem to e important. Speculations about is nature have been going on for literally thousands of years, and one odd finding that makes the problem puzzling is that it looks very much as if sleeping is not simply a matter of giving the body a rest. ``Rest'', in terms of muscle relaxation and so on, can be achieved by a brief period lying, or even sitting down. The body's tissues are self-repairing and self-restoring to a degree, and function best when more or less continuously active. In fact a basic amount of movement occurs during sleep which is specifically concerned with preventing muscle inactivity.

If it is not a question of resting the body, then perhaps it is the brain that needs resting? This might be a plausible hypothesis were it not for two factors. First the electroencephalograph (which is simply a device for recording the electrical activity of the brain by attaching electrodes to the scalp) shows that while there is a change in the pattern of activity during sleep, there is no evidence that the total amount of activity is any less. The second factor is more interesting and more fundamental. Some years ago an American psychiatrist named William Dement published experiments dealing with the recording of eye-movements during sleep. He showed that the average individual's sleep cycle is punctuated with peculiar bursts of eye-movements, some drifting and slow, others jerky and rapid. People woken during these periods of eye-movements generally reported that they had been dreaming. When woken at other times they reported no dreams. If one group of people were disturbed from their eye-movement sleep for several nights on end, and another group were disturbed for an equal period of time but when they were no exhibiting eye-movements, the first group began to show some personality disorders while the others seemed more or less unaffected. The implications of all this were that it was not the disturbance of sleep that mattered, but the disturbance of dreaming.
\subsection{Snake poison}
\label{sec-2-20}

How it came about that snakes manufactured poison is a mystery. Over the periods their saliva, a mild, digestive juice like our own, was converted into a poison that defies analysis even today. It was not forced upon them by the survival competition; they could have caught and lived on prey without using poison, just as the thousands of non-poisonous snakes still do. Poison to a snake is merely a luxury; it enables it to get its food with very little effort, no more effort than one bite. And why only snakes? Cats, for instance, would be greatly helped; no running fights with large, fierce rats or tussles with grown rabbits---just a bite and no more effort needed. In fact, it would be an assistance to all carnivores though it would be a two-edged weapon when they fought each other. But, of the vertebrates, unpredictable Nature selected only snakes (and one lizard). One wonders saliva into why Nature, with respect from that of others, as other on the blood.

In the conversion of saliva into poison, one might suppose that a fixed process took place. It did not; some snakes manufacture a poison different in every respect from that of others, as different as arsenic is from strychnine, and having different effects. One poison acts on the nerves, the other on the blood.

The makers of the nerve poison include the mambas and the cobras and their venom is called neurotoxic. Vipers (adders) and rattlesnakes manufacture the blood poison, which is known as haemolytic. Both poisons are unpleasant, but by far the more unpleasant is the blood poison. It is said that the nerve poison is the more primitive of the two, that the blood poison is, so to speak, a newer product from an improved formula. Be that as it may, the nerve poison does its business with man far more quickly than the blood poison. This, however, means nothing. Snakes did not acquire their poison for use against man but for use against prey such as rats and mice, and the effects on these of viperine poison is almost immediate.
\subsection{William S. Hart and the early ``Western'' film}
\label{sec-2-21}

William S. hart was, perhaps, the greatest of all Western stars, fro unlike Gary Cooper and John Wayne he appeared in nothing but Westerns. From 1914 to 1924 he was supreme and unchallenged. It was Hart who created the basic formula of the Western film, and devised the protagonist he played in every film he made, the good-had man, the accidental-noble outlaw, or the honest-but-framed cowboy, or the sheriff made suspect by vicious gossip; in short, the individual in conflict with himself and his frontier environment.

Unlike most of his contemporaries in Hollywood, Hart actually knew something of the old West. He had lived in it as a child when it was already disappearing, and his hero was firmly rooted in his memories and experiences, and in both the history and the mythology of the vanished frontier. And although no period or place in American history has been more absurdly romanticized, myth and reality did join hands in at least one arena, the conflict between the individual and encroaching civilization.

Men accustomed to struggling for survival against the elements and Indians were bewildered by politicians, bankers and businessmen, and unhorsed by fences, laws and alien taboos. Hart's good-bad man was always an outsider, always one of the disinherited, and if he found it necessary to shoot a sheriff or rob a bank along the way, his early audiences found it easy to understand and forgive, especially when it was Hart who, in the end, overcame the attacking Indians.

Audiences in the second decade of the twentieth century found it pleasant to escape to a time when life, though hard, was relatively simple. We still do; living in a world in which undeclared aggression, war, hypocrisy, chicanery, anarchy and impending immolation are part of our daily lives, we all want a code to live by.
\subsection{Knowledge and progress}
\label{sec-2-22}

Why does the idea of progress loom so large in the modern world? Surely progress of a particular kind is actually taking place around us and is becoming more and more manifest. Although mankind has undergone no general improvement in intelligence or morality, it has made extraordinary progress in the accumulation of knowledge. Knowledge began to increase as soon as the thoughts of one individual could be communicated to another by means of speech. With the invention of writing, a great advance was made, for knowledge could then be not only communicated but also stored. Libraries made education possible, and education in its turn added to libraries: the growth of knowledge followed a kind of compound interest law, which was greatly enhanced by the invention of printing. All this was comparatively slow until, with the coming of science, the tempo was suddenly raised. Then knowledge began to be accumulated according to a systematic plan. The trickle became a stream; the stream has now become a torrent. Moreover, as soon as new knowledge is acquired, it is now turned to practical account. What is called ``modern civilization'' is not the result of a balanced development of all man's nature. but of accumulated knowledge applied to practical life. The problem now facing humanity is: What is going to be done with all this knowledge? As is so often pointed out, knowledge is a two-edged weapon which can be used equally for good or evil. It is now being used indifferently for both. Could any spectacle, for instance, be more grimly whimsical than that of gunners ourselves very seriously what will happen if this twofold use of knowledge, with its ever-increasing power, continues.
\subsection{Bird flight}
\label{sec-2-23}

No two sorts of birds practise quite the same sort of flight; the varieties are infinite; but two classes may be roughly seen. Any shi that crosses the Pacific is accompanied for many days by the smaller albatross, Which may keep company with the vessel for an hour without visible or more than occasional movement of wing. The currents of air that the walls of the ship direct upwards, as well as in the line of its course, are enough to give the great bird with its immense wings sufficient sustenance and progress. The albatross is the king of the gliders, the class of fliers which harness the air to their purpose, but must yield to its opposition. In the contrary school, the duck is supreme. It comes nearer to the engines with which man has ``conquered'' the air, as he boasts. Duck, and like them the pigeons, are endowed with such-like muscles, that are a good part of the weight of the bird, and these will ply the short wings with such irresistible power that they can bore for long distances through an opposing gale before exhaustion follows. Their humbler followers, such as partridges, have a like power of strong propulsion, but soon tire. You may pick them up in utter exhaustion, if wind over the sea has driven them to a long journey. The swallow shares the virtues of both schools in highest measure. It tires not, nor does it boast of its power; but belongs to the air, travelling it may be six thousand miles to and from its northern nesting home, feeding its flown young as it flies, and slipping through we no longer take omens from their flight on this side and that; and even the most superstitious villagers no longer take off their hats to the magpie and wish it good-morning.
\subsection{Beauty}
\label{sec-2-24}

A young man sees a sunset and, unable to understand or to express the emotion that it rouses in him, concludes that it must be the gateway to world that lies beyond. It is difficult for any of us in moments of intense aesthetic experience to resist the suggestion that we are catching a glimpse of a light that shines down to us from a different realm of existence, different and, because the experience is intensely moving, in some way higher. And, though the gleams blind and dazzle, yet do they convey a hint of beauty and serenity greater than we have known or imagined. Greater too than we can describe; for language, which was invented to convey the meanings of this world, cannot readily be fitted to the uses of another.

That all great has this power of suggesting a world beyond is undeniable. In some moods, Nature shares it. There is no sky in June so blue that it does not point forward to a bluer, no sunset so beautiful that it does not waken the vision of a greater beauty, a vision which passes before it is fully glimpsed, and in passing leaves and indefinable longing and regret. But, if this world is not merely a bad joke, life a vulgar flare amid the cool radiance of the stars, and existence an empty laugh braying across the mysteries; if these intimations of a something behind and beyond are not evil humour born of indigestion, or whimsies sent by the devil to mock and madden us. if, in a word, beauty means something, yet we must not seek to interpret the meaning. If we glimpse the unutterable, it is unwise to try to utter it, nor should we seek to invest with significance that which we cannot grasp. Beauty in terms of our human meanings is meaningless.
\subsection{Non-auditory effects of noise}
\label{sec-2-25}

May people in industry and the Services, who have practical experience of noise, regard any investigation of this question as a waste of time; they are not prepared even to admit the possibility that noise affects people. On the other hand, those who dislike noise will sometimes use most inadequate evidence to support their pleas for a quieter society. This is a pity, because noise abatement really is a good cause, and it is likely to be discredited if it gets to be associated with had science.

One allegation often made is that noise produces mental illness. A recent article in a weekly newspaper, for instance, was headed with a striking illustration of a lady in a state of considerable distress, with the caption ``She was yet another victim, reduced to a screaming wreck''. On turning eagerly to the text, one learns that the lady was a typist who found the sound of office typewriters worried her more and more until eventually she had to go into a mental hospital. Now the snag in this sort of anecdote is of course that one merely a symptom? Another patient might equally well complain that her neighbours were combining to slander her and persecute her, and yet one might be cautious about believing this statement.

What is needed in case of noise is a study of large numbers of people living under noisy conditions, to discover whether they are mentally ill more often than other people are. Some time ago the United States Navy, for instance, examined a very large number of men working on aircraft carriers: the study was known as Project Anehin. It can be unpleasant to live even several miles from an aerodrome; if you think what it must be like to share the deck of a ship with several squadrons of jet aircraft, you will realize that a modern navy is a good place to study noise. But neither psychiatric interviews nor objective tests were able to show any effects upon these American sailors. This result merely confirms earlier American and British studies: if there is any effect of noise upon mental health, it must be so small that present methods of psychiatric diagnosis cannot find it. That does not prove that it does exist: but it does mean that noise is less dangerous than, say, being brought up in an orphanage---which really is mental health hazard.
\subsection{The past life of the earth}
\label{sec-2-26}

It is animals and plants which lived in or near water whose remains are most likely to be preserved, for one of the necessary conditions of preservation is quick burial, and it is only in the seas and rivers, and sometimes lakes, where mud and sit have been continuously deposited, that bodies and the can be rapidly covered over and preserved.

But even in the most favourable circumstances only a small fraction of the creatures that die are preserved in this way before decay sets in or, even more likely, before scavengers eat them. After all, all living creatures live by feeding on something else, whether it be plant or animal, dead or alive, and it is only by chance that such a fate is avoided. The remains of plants and animals that lived on land are much more rarely preserved, for there is seldom anything to cover them over. When you think of the innumerable birds that one sees flying bout, not to mention the equally numerous small animals like field mice and voles which you do not see, it is very rarely that one comes across a dead body, except, of course, on the roads. They decompose and are quickly destroyed by the weather or eaten by some other creature.

It is almost always due to some very special circumstances that traces of land animals survive, as by falling into inaccessible caves, or into an ice crevasse, like the Siberian mammoths, when the whole animal is sometimes preserved, as in a refrigerator. This is what happened to the famous Beresovka mammoth which was found preserved and in good condition. In his mouth were the remains of fir trees---the last meal that he had before he fell into the crevasse and broke his back. The mammoth has now just a suburb of Los Angeles. Apparently what happened was that water collected on these tar pits, and the bigger animals like the elephants ventured out on to the apparently firm surface to drink, and were promptly bogged in the tar. And then, when they were dead, the carnivores, like the sabre-toothed cats and the giant wolves, came out to feed and suffered exactly the same fate. There are also endless numbers of birds in the tar as well.
\subsection{The ``Vasa''}
\label{sec-2-27}

From the seventeenth-century empire of Sweden, the story of a galleon that sank at the start of her maiden voyage in 1628 must be one of the strangest tales of the sea. For nearly three and a half centuries she lay at the bottom of Stockholm harbour until her discovery in 1956. This was the Vasa, royal flagship of the great imperial fleet.

King Gustavus Adolphus, ``The Northern Hurricane'', then at the height of his military success in the Thirty Years' War, had dictated her measurements and armament. Triple gun-decks mounted sixty-four bronze cannon. She was intended to play a leading role in the growing might of Sweden.

As she was prepared of her maiden voyage on August 10, 1628, Stockholm was in a ferment. From the Skeppsbron and surrounding islands the people watched this thing of beauty begin to spread her sails and catch the wind. They had laboured for three years to produce this floating work of art; she was more richly carved and ornamented than any previous ship. The high stern castle was a riot of carved gods, demons, knights, kings, warriors, mermaids, cherubs; and zoomorphic animal shapes ablaze with rea and gold and blue, symbols of courage, power, and cruelty, were portrayed to stir the imaginations of the superstitious sailors of the day.

Then the cannons of the anchored warships thundered a salute to which the Vasa fired in reply. As the emerged from her drifting cloud of gun smoke with the water churned to foam beneath her bow, her flags colour, she presented a more majestic spectacle than Stockholmers had ever seen before. All gun-ports were open and the muzzles peeped wickedly from them.

As the wind freshened there came a sudden squall and the ship made a strange movement, listing to port. The Ordnance Officer ordered all the port cannon to be heaved to starboard to counteract the list, but the steepening angle of the decks increased. Then the sound of rumbling thunder reached the watchers on the shore, as cargo, ballast, ammunition and 400 people went sliding and crashing down to the port side of the steeply listing ship. The lower gun-ports were now below water and the inrush sealed the ship's fate. In that first glorious hour, the mighty Vasa, which was intended to rule the Baltic, sank with all flags flying-in the harbour of her birth.
\subsection{Patients and doctors}
\label{sec-2-28}

This is a sceptical age, but although our faith in many of the things in which our forefathers fervently believed has weakened, our confidence in the curative properties of the bottle of medicine remains the same a theirs. This modern faith in medicines is proved the fact that the annual drug bill of the Health Services is mounting to astronomical figures and shows no signs at present of ceasing to rise. The majority of the patients attending the medical out-patients departments of our hospitals feel that they have not received adequate treatment unless they are able to carry home with them some tangible remedy in the shape of a bottle of medicine, a box of pills, or a small jar of ointment, and the doctor in charge of the department is only too ready to provide them with these requirements. There is no quicker method of disposing of patients then by giving them what they are asking for, and since most medical men in the Health Services are overworked and have little time for offering time-consuming and little-appreciated advice on such subjects as diet, right living, and the need for abandoning bad habits etc., the bottle, the box, and the jar are almost always granted them.

Nor is it only the ignorant and ill-educated person who was such faith in the bottle of medicine. It is recounted of Thomas Carlyle that when him in his pocket what remained of a bottle of medicine formerly prescribed for an indisposition of Mrs. Carlyle's. Carlyle was entirely ignorant of what the bottle in his pocket contained, of the nature of the illness from which his friend was suffering, and of what had previously been wrong with his wife, but a medicine that had worked so well in one form of illness would surely be of equal benefit in another, and comforted by the thought of the help he was bringing to his friend, he hastened to Henry Taylor's house. History does not relate whether his friend accepted his medical help, but in all probability he did. The great advantage of taking medicine is that it makes no demands on the taker beyond that of putting up for a moment with a disgusting taste, and that is what all patients demand of their doctors---to be cured at no inconvenience to themselves.
\subsection{The hovercraft}
\label{sec-2-29}

Many strange new means of transport have been developed in our century, the strangest of them being perhaps the hovercraft. In 1953, a former electronics engineer in his fifties, Christopher Cockerell, who had turned to boat-building on the Norfolk Broads, suggested an idea on which he had been working for many years to the British Government and industrial circles. It was the idea of supporting a craft on a ``pad'', or cushion, of low-pressure air, ringed with a curtain of higher pressure air. Ever since, people have had difficulty in deciding whether the craft should be ranged among ships, planes, or land vehicles---for it is something in between a boat and an aircraft. As a shipbuilder, Cockerell was trying to find a solution to the problem of the wave resistance which wastes a good deal of a surface ship's power and limits its speed. His answer was to lift the vessel out of the water by a great number of ring-shaped air jets on the bottom of the craft. It ``flies'', therefore, but it cannot fly higher---its action depends on the surface, water or ground, over which it rides.

The first tests on the Solent in 1959 caused a sensation. The hovercraft travelled first over the water, then mounted the beach, climbed up the dunes, and sat down on a road. Later it crossed the Channel, riding smoothly over the waves, which presented no problem.

Since that time, various types of hovercraft have appeared and taken up regular service. The hovercraft is particularly useful in large areas with poor communications such as Africa or Australia; it can become a ``flying fruit-bowl'', carrying bananas from the plantations to the ports; giant hovercraft liners could span the Atlantic; and the railway of the future may well be the ``hovertrain'', riding on its air cushion over a single rail, which it never touches, at speeds up to 300 m.p.h.---the possibilities appear unlimited.
\subsection{Exploring the sea-floor}
\label{sec-2-30}

Our knowledge of the oceans a hundred years ago was confined to the two-dimensional shape of the sea surface and the hazards of navigation presented by the irregularities in depth of the shallow water close to the land. The open sea was deep and mysterious, and anyone who gave more than a passing thought to the bottom confines of the oceans probably assumed that the sea bad was flat. Sir James Clark Ross had obtained a sounding of over 2,400 fathoms in 1839, but it was not until of deep soundings was obtained in the Atlantic and the first samples were collected by dredging the bottom. Shortly after this the famous H. M. S. Challenger expedition established the study of the sea-floor as a subject worthy of the most qualified physicists and geologists. A burst of activity associated with the laying of submarine cables soon confirmed the challenger's observation that many parts of the ocean were two to there miles deep, and the existence of underwater features of considerable magnitude.

Today, enough soundings are available to enable a relief map of the Atlantic to be drawn and we know something of the great variety of the sea bed's topography. Since the sea covers the greater part of the earth's surface, it is quite reasonable to regard the sea floor as the basic form of the crust of the earth, with, superimposed upon, it the continents, together with the islands and other features of the oceans. The continents form rugged tablelands which stand nearly three miles above the floor of the open ocean. From the shore line, out a distance which may be anywhere from a few miles to a few hundred miles, runs the gentle slope of the continental shelf, geologically part of the continents. The real dividing line between continents and oceans occurs at the foot a steeper slope.

This continental slope usually starts at a place somewhere near the 100-fatheom mark and in the course of a few hundred miles reaches the true ocean floor at 2,500-3,500 fathoms. The slope averages about 1 in 30. but contains steep, probably vertical, cliffs, and gentle sediment-covered terraces, and near its lower reaches there is a long tailing-off which is almost certainly the result of material transported out to deep water after being eroded from the continental masses.
\subsection{The sculptor speaks}
\label{sec-2-31}

Appreciation of sculpture depends upon the ability to respond to form in there dimension. That is perhaps why sculpture has been described as the most difficult of all arts; certainly it is more difficult than the arts which involve appreciation of flat forms, shape in only two dimensions. Many more people are ``form-blind'' than colour-blind. The child learning to see, first distinguishes only two-dimensional shape; it cannot judge distances, depths. Later, for its personal safety and practical needs, it has to develop (partly by means of touch) the ability to judge roughly three-dimensonal distances. But having satisfied the requirements of practical necessity, most people go no further. Though they may attain considerable accuracy in the perception of flat from, they do no make the further. Though they may attain considerable accuracy in the perception of flat form, they do not make the further intellectual and emotional effort needed to comprehend form in its full spatial existence.

This is what the sculptor must do. He must strive continually to think of, and use, form in its full spatial completeness. He gets the solid shape, as it were, inside his head-he thinks of it, whatever its size, as if he were holding it completely enclosed in the hollow of his hand. He mentally visualizes a complex form from all round itself; he knows while he looks at one side what the other side is like, he identifies himself with its centre of gravity, its mass, its weight; he realizes its volume, as the space that the shape displaces in the air.

And the sensitive observer of sculpture must also learn to feel shape simply as shape, not as description or reminiscence. He must, for example, perceive an egg as a simple single solid shape, quite apart from its significance as food, or from the literary idea that it will become a bird. And so with solids such as a shell, a nut, a plum, a pear, a tadpole, a mushroom, a mountain peak, a kidney, a carrot, a tree-trunk, a bird, a bud, a lark, a ladybird, a bulrush, a bone. From these he can go on to appreciate more complex forms of combinations of several forms.
\subsection{Galileo reborn}
\label{sec-2-32}

In his own lifetime Galileo was the centre of violent controversy; but the scientific dust has long since settled, and today we can see even his famous clash with the Inquisition in something like its proper perspective. But, in contrast, it is only in modern times that Galileo has become a problem child for historians of science.

The old view of Galileo was delightfully uncomplicated. He was, above all, a man who experimented: who despised the prejudices and book learning of the Aristotelians, who put his questions to nature instead of to the ancients, and who drew his conclusions fearlessly. He had been the first to turn a telescope to the sky, and he had seen there evidence enough to overthrow Aristotle and Ptolemy together. He was the man who climbed the Leaning Tower of Pisa and dropped various weights from the top, who rolled balls down inclined planes, and then generalized the results of his many experiments into the famous law of free fall.

But a closer study of the evidence, supported by a deeper sense of the period, and particularly by a new consciousness of the philosophical undercurrents in the scientific revolution, has profoundly modified this view of Galileo. Today, although the old Galileo lives on in many popular writings, among historians of science a new and more sophisticated picture has emerged. At the same time our sympathy fro Galileo's opponents ahs grown somewhat. His telescopic observations are justly immortal; they aroused great interest at the time, they had important theoretical consequences, and they provided a striking demonstration of the potentialities hidden in instruments and apparatus. But can we blame those who looked and failed to see what Galileo saw, if we remember that to use a telescope at the limit of its powers calls for long experience and intimate familiarity with one's instrument? Was the philosopher who refused to look through Galileo's telescope more culpable than those who alleged that the spiral nebulae observed with Lord Rosse's great telescope in the eighteen-forties were scratches left by the grinder? We can perhaps forgive those who said the moons of Jupiter were produced by Galileo's spyglass if we recall that in his day, as for centuries before, curved glass was the popular contrivance for producing not truth but illusion, untruth; and if a single curved glass would distort nature, how much more would a pair of them?
\subsection{Education}
\label{sec-2-33}

Education is one of the key words of our time. A man without an education, many of us believe, is an unfortunate victim of adverse circumstances, deprived of one of the greatest twentieth-century opportunities. Convinced of the importance of education, modern states ``invest'' in institutions of learning to get back ``interest'' in the form of a large group of enlightened young men and women who are potential leaders. Education, with its cycles of instruction so carefully worked out, punctuated by textbooks---those purchasable wells of wisdom-what would civilization be like without its benefits?

So much is certain: that we would have doctors and preachers, lawyers and defendants, marriages and births---but our spiritual outlook would be different. We would lay less stress on ``facts and figures'' and more on a good memory, on applied psychology, and on the capacity of a man to get along with his fellow-citizens. If our educational system were fashioned after its bookless past we would have the most democratic form of ``college'' imaginable. Among tribal people all knowledge inherited by tradition is shared by all; it is taught to every member of the tribe so that in this respect everybody is equally equipped for life.

It is the ideal condition of the ``equal start'' which only our most progressive forms of modern education try to regain. In primitive cultures the obligation to seek and to receive the traditional instruction is binding to all. There are no ``illiterates''---if the term can be applied to peoples without a script---while our own compulsory school attendance became law in Germany in 1642, in France in 1806, and in England in 1876, and is still non-existent in a number of ``civilized'' nations. This shows how long it was before we deemed it necessary to make sure that all our children could share in the knowledge accumulated by the ``happy few'' during the past centuries.

Education in the wilderness is not a matter of monetary means. All are entitled to an equal start. There is none of the hurry which, in our society, often hampers the full development of a growing personality. There, a child grows up under the ever-present attention of his parent; therefore the jungles and the savannahs know of no ``juvenile delinquency''. No necessity of making a living away from home results in neglect of children, and no father is confronted with his inability to ``buy'' an education for his child.
\subsection{Adolescence}
\label{sec-2-34}

Parents are often upset when their children praise the homes of their friends and regard it as a slur on their own cooking, or cleaning, or furniture, and often are foolish enough to let the adolescents see that they are annoyed. They may even accuse them of disloyalty, or make some spiteful remark about the friends' parents. Such loss of dignity and descent into childish behaviour on the part to their parents about the place or people they visit. Before very long the parents will be complaining that the child is so secretive and never tells them anything, but they seldom realize that they have brought this on themselves. 

Disillusionment with the parents, however good and adequate they may be both as parents and as individuals, is to some degree inevitable. Most children have such a high ideal of their parents, unless the parents themselves have been unsatisfactory, that it can hardly hope to stand up to a realistic evaluation. Parents would be greatly surprised and deeply touched if they hope to stand up to a realistic evaluation. Parents would be greatly surprised and deeply touched if they realized how much belief their children usually have in their character and infallibility, and how much this faith means to a child. If parents were prepared for this adolescent reaction, and realized that it was a sign that the child was growing up and developing valuable powers of observation and independent judgment, they would not be so hurt, and therefore would not drive the child into opposition by resenting and resisting it.

The adolescent, with his passion for sincerity, always respects a parent who admits that he is wrong, or ignorant, or even that he has been unfair or unjust. What the child cannot forgive is the parent's refusal to admit these charges if the child knows them to be true.

Victorian parents believed that they kept their dignity by retreating behind an unreasoning authoritarian attitude; in fact they did nothing of the kind, but children were then too cowed to let them know how they really felt. Today we tend to go to the other extreme, but on the whole this is a healthier attitude both for the child and the parent. It is always wiser and safer to face up to reality, however painful it may be at the moment.
\subsection{Space odyssey}
\label{sec-2-35}

The Moon is likely to become the industrial hub of the Solar System, supplying the rocket fuels fro its ships, easily obtainable from the lunar rocks in the from of liquid oxygen. The reason lies in its gravity. Because the Moon has only an eightieth of the Earth's mass, it requires 97 per cent less energy to travel the quarter of a million miles from the Moon to Earth-orbit than the 200 mile-journey from Earth's surface into orbit!

This may sound fantastic, but it is easily calculated. To escape from the Earth in a rocket, one must travel at seven miles per second. The comparable speed from the Moon is only 1.5 miles per second. Because the gravity on the Moon's surface is only a sixth of Earth's (remember how easily the Apollo astronauts bounded along), it takes much less energy to accelerate to that 1.5 miles per second than it does on Earth. Moon-dwellers will be able to fly in space at only three per cent of the cost of similar journeys by their terrestrial dwellers will be able to fly in space at only three per cent of the cost of similar journeys by their terrestrial cousins.

Arthur C. Clark once suggested a revolutionary idea passes through three phases:

1 ``It's impossible---don't waste my time.''

2 ``It's possible, but not worth doing.''

3 ``I said it was a good idea all along.''

The idea of colonising Mars---a world 160 times more distant time the Moon---will move decisively from the second phase to the third, when a significant number of people are living permanently in space. Mars has an extraordinary fascination for would-be voyagers. America, Russia and Europe are filled with enthusiasts---many of them serious and senior scientists---who dream of sending people to it. Their aim is understandable. It is the one world in the Solar System that is most like the Earth. It is a world of red sandy deserts (hence its name---the Red Planet), cloudless skies, savage sandstorms, chasms wider than the Grand Canyon and at least one mountain more than twice as tall as Everest. It seems ideal for settlement.
\subsection{The cost of government}
\label{sec-2-36}

If a nation is essentially disunited, it is left to the government to hold it together. This increases the expense of government, and reduces correspondingly the amount of economic resources that could be used for developing the country. And it should not be forgotten how small those resources are in a poor and backward country. Where the cost of government is high, resources for development are correspondingly low.

This may be illustrated by comparing the position of a nation with that of a private business enterprise. An enterprise has to incur certain costs and expenses in order to stay in business. For our purposes, we are concerned only with one kind of cost---the cost of managing and administering the business. Such administrative overheads in a business are analogous to the cost of government in a nation. The administrative overheads of a business are low to the extent that everyone working in the business can be trusted to behave in a way that best promotes the interests of the firm. If they can each be trusted to take such responsibilities. and to exercise such initiative as falls within their sphere, then administrative overheads will be low. It will be low because it will be necessary to have only one man looking after each job, then the business will require armies of administrators, checkers, and foremen and administrative overheads will rise correspondingly. As administrative overheads rise, so the earnings of the business after meeting he expense of administration, will fall; and the business will have less money to distribute as dividends or invest directly in its future progress and development.

It is precisely the same with a nation. To the extent that the people can be relied upon to behave in a loyal and responsible manner, the government does not require armies of police and civil servants to keep them in order. But if a nation is disunited, the government cannot be sure that the actions of the people will be in the interests of the nation; and it will have to watch, check, and control the people accordingly. A disunited nation therefore has to incur unduly high costs of government.
\subsection{The process of aging}
\label{sec-2-37}

At the age of twelve years, the human body is at its most vigorous. It has yet to reach its full size and strength, and its owner his or her full intelligence; but at this age the likelihood of death is least. Earlier, we were infants and young children, and consequently more vulnerable; later, we shall undergo a progressive loss of our vigour and resistance which, though imperceptible at first, will finally become so steep that we can live no longer, however well we look after ourselves, and however well society, and our doctors, look after us. This decline in vigour with the passing of time is called ageing. It is one of the most unpleasant discoveries which we all make that we must decline in this way, that if we escape wars, accidents and disease we shall eventually ``die of old age'', and that this happens at a rate which differs little from person to person, so that there are heavy odds in favour of our dying between the ages of sixty-five and eighty. Some of us will die sooner, a few will live longer---on into a ninth or tenth decade. But the chances are against it, and there is a virtual limit on how long we can hope to remain alive, however lucky and robust we are.

Normal people tend to forget this process unless and until they are reminded of it. We are so familiar with the fact that man ages, that people have for years assumed that the process of losing vigour with time, of becoming more likely to die the older we get, was something self-evident, like the cooling of a hot kettle or the wearing-out of a pair of shoes. They have also assumed that all animals, and probably other organisms such as trees, or even the universe itself, must in the nature of things ``wear out''. Most animals we commonly observe do in fact age as we do, if given the chance to live long enough; and mechanical systems like a wound watch, or the sun, do in fact run out of energy in accordance with the second law of thermodynamics (whether the whole universe does so is a moot point at present). But these are not analogous to what happens when man ages. A run-down watch is still a watch and can be rewound. An old watch, by contrast, becomes so worn and unreliable that it eventually is not worth mending. But a watch could never repair itself---it does not consist of living parts, only of metal, which wears away by friction. We could, at one time, repair ourselves---well enough, at least, to overcome all but the most instantly fatal illnesses and accidents. Between twelve and eighty years we gradually lose this power; an illness which at twelve would knock us over, at eighty can knock us out, and another 700 for the survivors to be reduced by half again.
\subsection{Water and the traverller}
\label{sec-2-38}


Contamination of water supplies is usually due to poor sanitation close to water sources, sewage disposal into the sources themselves, leakage of sewage into distribution systems or contamination with industrial or farm waste. Even if a piped water supply is safe at its source, it is not always safe by the time it reaches the tap. Intermittent tap-water supplies should be regarded as particularly suspect.

Travellers on short trips to areas with water supplies of uncertain quality should avoid drinking tap-water, or untreated water from any other source. It is best to hot drinks, bottled or canned drinks of well-known brand names---international standards of water treatment are usually followed at bottling plants. Carbonated drinks are acidic, and slightly safer. Make sure that all bottles are opened in your presence, and that their rims are clean and dry.

Boiling is always a good way of treating water. Some hotels supply boiled water on request and this can be used for drinking, or for brushing teeth. Portable boiling elements that can boil small quantities of water are useful when the right voltage of electricity is available. Refuse politely any cold drink from an unknown source.

Ice is only as safe as the water from which it is made, and should not be put in drinks unless it is known to be safe. Drink can be cooled by placing them on ice tather than adding ice to them.

Alcohol may be a medical disinfectant, but should not be relied upon to sterilize water. Ethanol is more effective at a concentration of 50-70 per cent; below 20 per cent, its bactericidal action is negligible. Spirits labelled 95 proof contain only about 47 per cent alcohol. Beware of methylated alcohol, which is very poisonous, and should never be added to drinking water.

If no other safe supply can be obtained, tap water that is too hot to touch can be left to cool and is generally safe to drink. Those planning a trip to remote areas, or intending to live in countries where drinking water is not readily available, should know about the various possible methods for making water safe.
\subsection{What every writer wants}
\label{sec-2-39}

I have known very few writers, but those I have known, and whom I respect, confess at once that they have little idea where they the are going when they first set pen to paper. They have a character, perhaps two; they are in that condition of eager discomfort which passes for inspiration; all admit radical changes of destination once the journey has begun; one, to my certain knowledge, spent nine months on a novel about Kashmir, then reset the whole thing in the Scottish Highlands. I never heard of anyone making a ``skeleton'', as we were taught at school. In the breaking and remaking, in the timing, interweaving, beginning afresh, the writer comes to discern things in his material which were not consciously in his mind when he began. This organic process, often leading to moments of extraordinary self-discovery, is of an indescribable fascination. A blurred image appears; he adds a brushstroke and another, and it is gone; but something was there, and he will not rest till he has captured it. Sometimes the yeast within a writer outlives a book he has written. I have heard of writers who read nothing but their own books; like adolescents they stand before the mirror, and still cannot fathom the exact outline of the vision before them. For the same reason, writers talk interminably about their own books, winkling out hidden meanings, super-imposing new ones, begging response from those around them. Of course a writer doing this is misunderstood: he might as well try to explain a crime or a love affair. He is also, incidentally, an unforgivable bore.

This temptation to cover the distance between himself and the reader, to study his image in the sight of those who do not know him, can be his undoing: he has begun to write to please.

A young English writer made the pertinent observation a year or two back that the talent goes into the first draft, and the art into the drafts that follow. For this reason also the writer, like any other artist, has no resting place, no crowd or movement in which he may take comfort, no judgment from outside which can replace the judgment from within. A writer makes order out of the anarchy of his heart; he submits himself to a more ruthless discipline than any critic dreamed of, and when he flirts with fame, he is taking time off from living with himself, from the search for what his world contains at its inmost point.
\subsection{Waves}
\label{sec-2-40}

Waves are the children of the struggle between ocean and atmosphere, the ongoing signatures of infinity. Rays from the sun excite and energize the atmosphere of the earth, awakening it to flow, to movement, to rhythm, to life. The wind then speaks the message of the sun to the sea and the sea transmits it on through waves---an ancient, exquisite, powerful message.

These ocean waves are among the earth's most complicated natural phenomena. The basic features include a crest (the highest point of the wave), a trough (the lowest point), a height (the vertical distance from the trough to the crest), a wave length (the horizontal distance between two wave crests), and a period (which is the time it takes a wave crest to travel one wave length).

Although an ocean wave gives the impression of a wall of water moving in your direction, in actuality waves move through the water leaving the water about where it was. If the water was moving with the wave, the ocean and everything on it would be racing in to the shore with obviously catastrophic results.

An ocean wave passing through deep water causes a particle on the surface to move in a roughly circular orbit, drawing the particle first towards the advancing wave, then up into the wave, then forward with it and then---as the wave leaves the particles behind---back to its starting point again.

From both maturity to death, a wave is subject to the same laws as any other ``living'' thing. For a time it assumes a miraculous individuality that, in the end, is reabsorbed into the great ocean of life.

The undulating waves of the open sea are generated by three natural causes: wind, earth movements or tremors, and the gravitational pull of the moon and the sun. Once waves have been generated, gravity is the force that drives them in a continual attempt to restore the ocean surface to a flat plain.
\subsection{Training elephants}
\label{sec-2-41}

Two main techniques have been used for training elephants, which we may respectively the tough and the gentle. The former method simply consists of setting an elephant to work and beating him until he does what is expected of him. Apart from moral considerations this is a stupid method of training, for it produces a resentful animal who at a later stage may well turn man-killer. The gentle method requires more patience in the early stages, but produces a cheerful, good-tempered elephant who will give many years of loyal service.

The first essential in elephant training is to assign to the animal a single mahout who will be entirely responsible for the job. Elephants like to have one master just as dogs do, and are capable of a considerable degree of personal affection. There are even stories of half-trained elephant calves who have refused to feed and pined to death when by some unavoidable circumstance they have been deprived of their own trainer. Such extreme cases must probably be taken with a grain of salt, but they do underline the general principle that the relationship between elephant and mahout is the key to successful training. 

The most economical age to capture an elephant for training is between fifteen and twenty years, for it is then almost ready to undertake heavy work and can begin to earn its keep straight away. But animals of this age do not easily become subservient to man, and a very time man, and a very firm hand must be employed in the early stages. The captive elephant, still roped to a tree, plunges and screams every time a man approaches, and for several days will probably refuse all food through anger and fear. Sometimes a tame elephant is tethered nearby to give the wild one confidence, and in most cases the captive gradually quietens down and begins to accept its food. The next stage is to get the elephant to the training establishment, a ticklish business which is achieved with the aid of two tame elephants roped to the captive on either side.

When several elephants are being trained at one time, it is customary for the new arrival to be placed between the stalls of two captives whose training is already well advanced. It is then left completely undisturbed with plenty of food and water so that it can absorb the atmosphere of its new home and see that nothing particularly alarming is happening to its companions. When it is eating normally, its own training begins. The trainer stands in front of the elephant holding a long stick with a sharp metal point. Two assistants, mounted on tame elephants, control the captive from either side, while others rub their hands over his skin to the accompaniment of a monotonous and soothing chant. This is supposed to induce pleasurable sensations in the elephant, and its effects are reinforced by the use of endearing epithets, such as `ho! my son', or ``ho! my father'', or ``ho! my mother'', according to the age and sex of the captive. The elephant is not immediately susceptible to such blandishments, however, and usually lashes fiercely with its trunk in all directions. These movements are controlled by the trainer with the metal-pointed stick, and the trunk eventually becomes so sore that the elephant curls it up and seldom afterwards uses it for offensive purposes.
\subsection{Recording an earthquake}
\label{sec-2-42}

An earthquake comes like a thief in the night, without warning. It was necessary, therefore, to invent instruments that neither slumbered nor slept. Some devices were quite simple. One, for instance, consisted of rods of various lengths and thicknesses with would stand up end like ninepins. When a shock came, it shook the rigid table upon which these stood. If it were gentle, only the more unstable rods fell. If it were severe, they all fell. Thus the rods, by falling, and by the direction in which they fell, recorded for the severe, they all fell. Thus the rods, by falling, and by the direction in which they fell, recorded for the slumbering scientist the strength of a shock that was too weak to waken him, and the direction from which it came.

But instruments far more deliecate than that were needed if any really serious advance was to be made. The ideal to be aimed at was to devise an instrument that could record with a pen on paper, the movements of the ground or of the table as the quake passed by. While I write my pen moves, but the paper keeps still. With practice, no doubt, I could in time learn to write by holding the pen still while the paper moved. That sounds a silly suggestion, but that was precisely the idea adopted in some of the early instruments (seismometers) for recording earthquake waves. But when table, penholder and paper are all moving, how is it possible to write legibly? The key to a solution of that problem lay in an everyday observation. Why does a person standing in a bus or train tend to fall when a sudden start is made? It is because his feet move on , but his head stays still. A simple experiment will help us a little further. Tie a heavy weight at the end of a long piece of string. With the hand to and fro and around but not up and string so that the weight nearly touches the ground. Now move the hand to and fro and around but not up and down. It will be found that the weight a piece of string. With the hand held high in the air, hold the string so that the weight nearly touches the ground. Now move the hand to and fro and around but not up and down. It will be found that ten weight moves but slightly or not at all. Imagine an earthquake shock shaking the floor, the paper, you and your hand. In the midst of all this movement, the weight and the pen would be still. But as the paper moved from side to side under the pen point, its movement would be recorded in ink upon its surface. It was upon this principle that the first instruments were made, but while the drum was being shaken, the line that the pen was drawing wriggled from side to side. The apparatus thus described, however, records only the horizontal component of the wave movement, which is, in fact, much more complicated. If we could actually see the path described by a particle, such as a sand grain in the rock, it would be more like that of a bluebottle path described by a particle, such as a sand grain in the rock, it would be more like that of a bluebottle buzzing round the room; it would be up and down, to and fro and from side to side. Instruments have been devised and can be so placed that all three elements can be recorded in different graphs.

When the instrument is situated at more than 700 miles from the earthquake centre, the graphic record shows three waves arriving one after at short intervals. The first records the arrival of longitudinal vibrations. The second marks the arrival of transverse vibrations which travel more slowly and arrive several minutes after the first. These two have travelled through the earth. It was from the study of these that so much was learnt about the interior of the earth. The third, or main. The third, or main wave, is the slowest and has travelled round the earth through the surface rocks.
\subsection{Are there strangers in space?}
\label{sec-2-43}

We must conclude from the work of those who have studied the origin of life, that given a planet only approximately like our own, life is almost certain to start. Of all the planets in our solar system, we ware now pretty certain the Earth is the only one on which life can survive. Mars is too dry and poor in oxygen, Venus far too hot, and so is Mercury, and the outer planets have temperatures near absolute zero and hydrogen-dominated atmospheres. But other suns, start as the astronomers call them, are bound to have planets like our own, and as is the number of stars in the universe is so vast, this possibility becomes virtual certainty. There are one hundred thousand million starts in our own Milky Way alone, and then there are exist is now estimated at about 300 million million.

Although perhaps only 1 per cent of the life that has started somewhere will develop into highly complex and intelligent patterns, so vast is the number of planets, that intelligent life is bound to be a natural part of the universe.

If then we are so certain that other intelligent life exists in the universe, why have we had no visitors from outer space yet? First of all, they may have come to this planet of ours thousands or millions of years ago, and found our then prevailing primitive state completely uninteresting to their own advanced knowledge. Professor Ronald Bracewell, a leading American radio astronomer, argued in Nature that such a superior civilization, on a visit to our own solar system, may have left an automatic messenger behind to await the possible awakening of an advanced civilization. Such a messenger, receiving our radio and television signals, might well re-transmit them back to its home-planet, although what impression any other civilization would thus get from us is best left unsaid.

But here we come up against the most difficult of all obstacles to contact with people on other planets---the astronomical distances which separate us. As a reasonable guess, they might, on an average, be 100 light years away. (A light year is the distance which light travels at 186,000 miles per second in one year, namely 6 million million miles.) Radio waves also travel at the speed of light, and assuming such an automatic messenger picked up our first broadcasts of the 1920's, the message to its home planet is barely halfway there. Similarly, our own present primitive chemical rockets, though good enough to orbit men, have no chance of transporting us to the nearest other star, four light years away, let alone distances of tens or hundreds of light years.

Fortunately, there is a ``uniquely rational way'' for us to communicate with other intelligent beings, as Walter Sullivan has put it in his excellent book, We Are not Alone. This depends on the precise radio frequency of the 21-cm wavelength, or 1420 megacycles per second. It is the natural frequency of emission of the hydrogen atoms in space and was discovered by us in 1951; it must be known to any kind of radio astronomer in the universe.

Once the existence of this wave-length had been discovered, it was not long before its use as the uniquely recognizable broadcasting frequency for interstellar communication was suggested. Without something of this kind, searching for intelligences on other planets would be like trying to meet a friend in London without a pre-arranged rendezvous and absurdly wandering the streets in the hope of a chance encounter.
\subsection{Patterns of culture}
\label{sec-2-44}

Custom has not commonly been regarded as a subject of great moment. The inner workings of our won brains we feel to be uniquely worthy of investigation, but custom, we have a way of thinking, is behaviour at its most commonplace. As a matter of fact, it is the other way around. Traditional custom, taken the world over, is a mass of detailed behaviour more astonishing than what any one person can ever evolve in individual actions, no matter how aberrant. Yet that is a rather trivial aspect of the matter. The fact of first-rate importance is the predominant role that custom plays in experience and in belief, and the very great varieties it may manifest.

No man ever looks at the world with pristine eyes. He sees it edited by a definite set of customs and institutions and ways of thinking. Even in his philosophical probing he cannot go behind these stereotypes; his very concepts of the true and the false will still have reference to his particular traditional customs. John Dewey has said in all seriousness that the part played by custom in shaping the behaviour of the individual, as against any way in which he can affect traditional custom, is as the proportion of the total vocabulary of his mother tongue against those words of his own baby talk that are taken up into the vernacular of his family. When one seriously studies the social orders that have had the opportunity to develop autonomously, the figure becomes no more than an exact and matter-of-fact observation. The life history handed down in his community. From the moment of his birth, the customs into which he is born shape his experience and behaviour. By the time he can talk, he is the little creature of his culture, and by the time he is grown and able to take part in its activities, its habits are his habits, its beliefs his beliefs, its impossibilities his impossibilities. Every child that is born into his group will share them with him, and no child born into one on the opposite side of the globe can ever achieve the thousandth part. There is no social problem it is more incumbent upon us to understand than this of the role of custom. Until we are intelligent as to its laws and varieties, the main complicating facts of human life must remain unintelligible.

The study of custom can be profitable only after certain preliminary propositions have been accepted, and some of these propositions have been violently opposed. In the first place, any scientific study requires that there be no preferential weighting of one or another of the items in the series it selects for its consideration. In all the less controversial fields, like the study of cacti or termites or the mature of nebulae, the necessary method of study is to group the relevant material and to take note of all possible variant forms and conditions. In this way, we have learned all that we know of the laws of astronomy, or of the habits of the social insects, let us say. It is only in the relevant material and to take note of all possible variant forms and conditions. In this way, we have learned all that we know of the laws of astronomy, or of the habits of the social insects, let us say. It is only in the study of man himself that the major social sciences have substituted the study of one local variation, that of Western civilization.

Anthropology was by definition impossible, as long as these distinctions between ourselves and the primitive, ourselves and the barbarian, ourselves and the pagan, held sway over people's minds. It was necessary first to arrive at that degree of sophistication where we no longer set our own belief against our neighbour's superstition. It was necessary to recognize that these institutions which are based on the same premises, let us say the supernatural, must be considered together, our own among the rest.
\subsection{Of men and galaxies}
\label{sec-2-45}

In man's early days. competition with other creatures must have been critical. But this phase of our development is now finished. Indeed, we lack practice and experience nowadays in dealing with primitive conditions. I am sure that, without modern weapons, I would make a very poor show of disputing the ownership of a cave with a bear, and in this I do not think that I stand alone. The last creature to compete with man was the mosquito. But even the mosquito has been subdued by attention to drainage and by chemical sprays.

Competition between our selves, person against person, community against community, still persists, however; and it is as fierce as it ever was.

But the competition of man against man is not the simple process envisioned in biology. It is not a simple competition for a fixed amount of food determined by the physical environment, because the environment that determines our evolution is no longer essentially physical. Our environment is chiefly conditoned by the things we believe. Morocco and California are bits of the Earth in very similar latitudes, both on the west coasts of continents with similar climates, and probably with rather similar natural resources. Yet their present development is wholly different, not so much because of different people wish to emphasize. The most important factor in our environment is the state of our own minds.

It is well known that where the white man has invaded a primitive culture, the most destructive effects have come not from physical weapons but from ideas. Ideas are dangerous. The Holy Office knew this full well when it caused heretics to be burned in days gone by. Indeed, the concept of free speech only exists in our modern society because when you are inside a community, you are conditioned by the conventions of the community to such a degree that it is very difficult to conceive of anything really destructive. It is only someone looking on from outside that can inject the dangerous thoughts. I do not doubt that it would be possible to inject ideas into the modern world that would utterly destroy us. I would like to give you an example, but fortunately I cannot do so. Perhaps it will suffice to mention the unclear bomb. Of making the effect on a reasonably advanced technological society, one that still does not possess the bomb, of making it aware of the possibility, of supplying sufficient details to enable the thing to be constructed. Twenty or thirty pages of information handed to any of the major world powers around the year 1925 would have been sufficient to change the course of world history. It is a strange thought, but I believe a correct one, that twenty or thirty pages of ideas and information would be capable of turning the present-day world upside down, or even destroying it. I have often tried to conceive of what those pages might contain, but of course outside the particular patterns that our brains are conditioned to, or, to be more accurate, we can think only a very little way outside, and then only if we are very original.
\subsection{Hobbies}
\label{sec-2-46}

A gifted American psychologist has said, ``Worry is a spasm of the emotion; the mind catches hold of something and will not let it go.'' It is useless to argue with the mind in this condition. The stronger the will, the more futile the task. One can only gently insinuate something else into its convulsive grasp. And if this something else is rightly chosen, if it really attended by the illumination of another field of interest, gradually, and often quite swiftly, the old undue grip relaxes and the process of recuperation and repair begins.

The cultivation of a hobby and new forms of interest is therefore a policy of the first importance to a public man. But this is not a business that can be undertaken in a day or swiftly improvised by a mere command of the will. The growth of alternative mental interests is a long process. The seeds must by carefully chosen; they must fall on good ground; they must be sedulously tended, if the vivifying fruits are to be at hand when needed.

To be really happy and really safe, one ought to have at least two or three hobbies, and they must all be real. It is no use starting late in life to say: ``I will take an interest in this or that.'' Such an attempt only aggravates the strain of mental effort. A man may acquire great knowledge of topics unconnected with his daily work, and yet get hardly any benefit or relief. It is no use doing what you like; you have got to like what you do. Broadly speaking, human beings may be divided into three classes: those who are toiled to death, those who are worried to hard week's sweat and effort, the chance of playing a game of football or baseball or Saturday afternoon. It is no use inviting the politician or the professional or business man, who has beer working or worrying about serious things for six days, to work or worry about trifling things at the weekend.

As for the unfortunate people who can command everything they want, who can gratify every caprice and lay their hands on almost every object of desire---for them a new pleasure, a new excitement if only an additional satiation. In vain they rush frantically round from place to place, trying to escape from avenging boredom by mere clatter and motion. For them discipline in one form or another is the most hopeful path. 

It may also be said that rational, industrious, useful human being are divided into two classes: first, one. Of these the former are the majority. They have their compensations. The long hours in the office or the factory bring with them as their reward, not only the means of sustenance, but a keen appetite for pleasure even in its simplest and most modest forms. But Fortune's of sustenance, but a keen appetite for pleasure even in its simplest and modest forms. But Fortune's favoured children belong to the second class. Their life is a natural harmony. For them the working hours are never long enough. Each day is a holiday, and ordinary holidays, when they come, are grudged as enforced as enforced interruptions in an absorbing vocation. Yet to both classes, the need of an alternative outlook, of a change of atmosphere, of a diversion of effort, is essential. Indeed, it may well be that those work is their pleasure are those who and most need the means of banishing it at intervals from their minds.
\subsection{The great escape}
\label{sec-2-47}

Economy is one powerful motive for camping, since after the initial outlay upon equipment, or through hiring it, the total expense can be far less than the cost of hotels. But, contrary to a popular assumption, it is far from being the only one, or even the greatest. The man who manoeuvres carelessly into his twenty pounds' worth of space at one of Europe's myriad permanent sites may find himself bumping a Bentley. More likely, Ford Escort will be hub to hub with Renault or Mercedes, but rarely with bicycles made for two.

That the equipment of modern camping becomes yearly more sophisticated is an entertaining paradox for the cynic, a brighter promise for the hopeful traveler who has sworn to get away from it all. It also provides-and some student sociologist might care to base his thesis upon the phenomenon---an escape of another kind. The modern traveller is often a man who dislikes the Splendide and the Bellavista, not because he cannot afford, or shuns their material comforts. but because he is afford of them. Affluent he may be, but he is by no means sure what to tip the doorman or the chambermaid. Master in his own house, he has little idea of when to say boo to a maitre d'hotel.

From all such fears camping releases him. Granted, a snobbery of camping itself, based upon equipment and techniques, already exists; but it is of a kind that, if he meets it, he can readily understand and deal with. There is no superior ``they'' in the shape of managements and hotel hierarchies to darken his holiday days.

To such motives, yet another must be added. The contemporary phenomenon of car worship is to be explained not least by the sense of independence and freedom that ownership entails. To this pleasure camping gives an exquisite refinement.

From one's own front door to home or foreign hills or sands and back again, everything is to hand. Not only are the means of arriving at the holiday paradise entirely within one's own command and keeping, but the means of escape from holiday hel (if the beach proves too crowded, the local weather too inclement) are there, outside---or, as likely, part of---the tent.

Idealists have objected to the package tour, that the traveller abroad thereby denies himself the opportunity of getting to know the people of the country visited. Insularity and self-containment, it is argued, go hand in hand. The opinion does not survive experience of a popular Continental camping place. Holiday hotels tend to cater for one nationality of visitors especially, sometimes exclusively. Camping sites, by contrast, are highly cosmopolitan. Granted, a preponderance of Germans is a characteristic that seems common to most Mediterranean sites; but as yet there is no overwhelmingly specialized patronage. Notices forbidding the open-air drying of clothes, or the use of water points for car washing, or those inviting ``our camping friends'' to a dance or a boat trip are printed not only in French or Italian or Spanish, but also in English, German and Dutch. At meal times the odour of sauerkraut vies with that of garlic. The Frenchman's breakfast coffee competes with the Englishman's bacon and eggs.

Whether the remarkable growth of organized camping means the eventual death of the more independent kind is hard to say. Municipalities naturally want to secure the campers' site fees and other custom. Police are wary of itinerants who cannot be traced to a recognized camp boundary or to four walls. But most probably it will all depend upon campers themselves: how many heath fires they cause; how much litter they leave; in short, whether or not they wholly alienate landowners and those who live in the countryside. Only good scouting is likely to preserve the freedoms so dear to the heart of the eternal Boy Scout.
\subsection{Planning a share portfolio}
\label{sec-2-48}

There is no shortage of tipsters around offering ``get-rich-quick'' opportunities. But if you are a serious private investor, leave the Las Vegas mentality to those with money to fritter. The serious investor needs a proper ``portfolio''---a well-planned selection of investments, with a definite structure and a clear aim. But exactly how does a newcomer to the stock market go about achieving that?

Well, if you go to five reputable stock brokers and ask them what you should do with your money, you're likely to get five different answers, ---even if you give all the relevant information about your age age, family, finances and what you want from your investments. Moral? There is no one ``right'' way to structure a portfolio. However, there are undoubtedly some wrong ways, and you can be sure that none of our five advisers would have suggested sinking all (or perhaps any) of your money into Periwigs*.

So what should you do? We'll assume that you have sorted out the basics---like mortgages, pensions, insurance and access to sufficient cash reserves. You should then establish your own individual aims. These are partly a matter of personal circumstances, partly a matter of psychology.

For instance, if you are older you have less time to recover from any major losses, and you may well wish to boost your pension income. So preserving your capital and generating extra income are your main priorities. In this case, you'd probably construct a portfolio with some shares (but not high risk ones), along with gilts, cash deposits, and perhaps convertibles or the income shares of split capital investment trusts.

If you are younger, and in a solid financial position, you may decide to take an aggressive approach---but only if you're blessed with a sanguine disposition and won't suffer sleepless nights over share prices. If portfolio, alongside your more pedestrian in vestments. Once you have decided on your investment aims, you can then decide where to put your money. The golden rule here is spread your risk---if you put all of your money into Periwigs International, you're setting yourself up as a hostage to fortune.

-> Periwigs is the name of a fictitious company.

\end{document}
